\subsection{Návrh řešení autonomní rekalibrace proudové tabulky}
Při běžné operaci robota můžeme sledovat polohy a rychlosti jednotlivých kloubů pro které se pohyb vyhlazuje pomocí proudové kalibrační tabulky. Sběrem vhodných bodů pro aktualizaci CCT se zabývá sekce \nameref{section:Analýza záznamu jednoho kloubu}.
\par
Kompenzační proud pohonu pro jeden směr pohybu je momentálně popsán dvěma kalibračními tabulkami (2 polohy kloubu $Ids$), kde každá tabulka popisuje 3D povrch. Nový způsob ukládání těchto závislostí je pomocí 4D NURBS nadpovrchu, což je čtyřrozměrný prostor, který již obsahuje kompletní informaci o kompenzačním proudu pro libovolné polohy všech tří kloubů $C_{Arc}, Prop, Ids$ v rámci jejich stanoveného rozsahu.
\par
Uvažujme tedy kompenzační tabulku popsanou 4D nadpovrchem. Vhodně konstruovaný algoritmus aktivně sleduje pohyby manipulátoru a vyhovující hodnoty kompenzačního proudu uloží pro účely aktualizace CCT. Na základě uložených hodnot můžeme poupravit 4D nadpovrch tak, aby na těchto místech byl kompenzační proud popsán přesněji.
\par
Kroky výsledného algoritmu robota pro jednu relaci by mohly vypadat nějak takto:
\begin{enumerate}
    \item Načtení a vytvoření kalibrační tabulky ve formě NURBS 4D nadpovrchu --- tj. provedení NURBS interpolace \todo{rozvést proč interpolace}
    \item \label{item:průběh práce} V průběhu práce s robotem se za běhu
          \begin{enumerate}
              \item z nadpovrchu vypočítá kompenzační proud na základě aktuálních poloh kloubů $C_{Arc}, Prop, Ids$.
              \item nasbírají vhodné body pro budoucí aktualizaci kompenzační tabulky.
          \end{enumerate}
    \item \label{item:konec relace} Na konci relace na základě nasbíraných bodů proběhne přepočtení a následné uložení kompenzační tabulky  --- tj. provedení NURBS aproximace \todo{rozvést proč aproximace viz článek}
\end{enumerate}
V této práci postupně projdeme všechny tyto zmíněné kroky.