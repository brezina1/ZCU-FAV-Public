\subsection{Návrh řešení autonomní rekalibrace proudové tabulky}
% Při běžné operaci robota můžeme sledovat polohy a~rychlosti jednotlivých kloubů --- sběrem vhodných bodů pro aktualizaci CCT se později zabývá sekce \nameref{section:Analýza záznamu jednoho kloubu}.
% \par
Kompenzační proud pohonu pro jeden směr pohybu je momentálně popsán dvěma kalibračními tabulkami (2 polohy kloubu $Ids$), přičemž se jedná o~soubor diskrétních dat na předem definované mřížce, kde každou tabulku lze reprezentovat 3D povrchem. Novým způsobem reprezentace těchto závislostí je pomocí 4D NURBS nadpovrchu, což je čtyřrozměrný prostor, který již obsahuje kompletní informaci o~kompenzačním proudu pro libovolné polohy všech tří kloubů $C_{Arc}, Prop, Ids$ v~rámci jejich stanoveného rozsahu ve formě spojité, hladké funkce.
\par
Uvažujme tedy kompenzační tabulku popsanou 4D nadpovrchem. Vhodně konstruo\-vaný algoritmus aktivně sleduje pohyby manipulátoru a~vyhovující hodnoty kompenzačního proudu uloží pro účely aktualizace CCT. Na základě uložených hodnot můžeme poupravit 4D nadpovrch tak, aby na těchto místech byl kompenzační proud popsán přesněji.
\par
Kroky výsledného algoritmu robota (kterými se tato práce zabývá) pro jednu relaci by mohly tedy vypadat nějak takto:
\begin{enumerate}
    \item Načtení a~vytvoření (= provedení NURBS interpolace) kalibrační tabulky ve tvaru NURBS 4D nadpovrchu. Kalibrační tabulka je uložena v podobě diskrétních bodů a pro dosažení optimálního výkonu je nutné, aby řídící systém dokázal spočítat kompenzační proud i mezi těmito body. K tomu právě slouží interpolace, která "převede" diskrétní body na hladkou a spojitou funkci.
    \item \label{item:průběh práce} V~průběhu práce s~robotem se za běhu
          \begin{enumerate}
              \item z~nadpovrchu vypočítá kompenzační proud na základě aktuálních poloh kloubů $C_{Arc}, Prop, Ids$.
              \item nasbírají vhodné body pro budoucí aktualizaci kompenzační tabulky.
          \end{enumerate}
    \item \label{item:konec relace} Na konci relace na základě nasbíraných bodů proběhne přepočtení (= provedení NURBS aproximace) a~následné uložení kompenzační tabulky. Aproximace slouží právě ke sloučení starých a nových dat kompenzační tabulky vhodnou, hladkou, spojitou funkcí. Tvar výsledné funkce lze přizpůsobit pomocí mnoha parametrů, jako jsou například: stupeň bázových funkcí, počet řídících bodů a váhy jednotlivých aproximačních bodů. Nakonec se výsledná funkce ve vhodných bodech vyhodnotí, abychom opět získali množinu diskrétních bodů.
\end{enumerate}
% V této práci postupně projdeme všechny tyto zmíněné kroky.