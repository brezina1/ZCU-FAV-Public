\subsubsection{Analýza testu odchylky proudu}
Tato sekce se zabývá záznamem číslo \ref{item:analýza soubor test proudu}. Záznam je tvořen vícero soubory, každý odpovídající
jednomu pohybu kloubu $C_{Arc}$ pro fixní polohu $Prop$ a $Ids$. Stejně jako v sekci \nameref{section:Analýza záznamu jednoho kloubu} se budeme zajímat o pomalý pohyb s konstantní rychlostí, vyhovující hodnoty pro jednotlivé typy trajektorií jsou zobrazeny na obrázcích č.~\ref{fig: trajektorie kloubu CArc dopředného pohybu Ids=0}, \ref{fig: trajektorie kloubu CArc dopředného pohybu Ids=1}, \ref{fig: trajektorie kloubu CArc zpětného pohybu Ids=0}, \ref{fig: trajektorie kloubu CArc zpětného pohybu Ids=1}.\par
Ze záznamu jsme vypočetli nový kompenzační proud kalibrační tabulky $I_{CCT}^{new}$ pomocí vzorečku:
\begin{align}
    I_{CCT}^{new} = I_{CCT} + I_{set} - I_{measured}
\end{align}
kde $I_{CTT}$ značí kompenzační proud z CCT, $I_{set}$ výstup regulátoru a $I_{measured}$ je měřený proud tekoucí pohonem. \par
Takto vypočtené hodnoty jsou zobrazeny na obrázcích č.~\ref{fig: Vypočtený kompenzační proud ze záznamu testovací trajektorie pro dopředný pohyb, Ids = 0 m}, \ref{fig: Vypočtený kompenzační proud ze záznamu testovací trajektorie pro dopředný pohyb, Ids = 1 m}, \ref{fig: Vypočtený kompenzační proud ze záznamu testovací trajektorie pro zpětný pohyb, Ids = 0 m}, \ref{fig: Vypočtený kompenzační proud ze záznamu testovací trajektorie pro zpětný pohyb, Ids = 1 m}. Jak lze vidět, výsledky jsou velmi zašuměné, přesto můžeme vypozorovat, že zašuměná data přibližně opisují kalibrační tabulku.
\paragraph{Zpracování zašuměných dat}
Na zašuměná data zkusíme aplikovat 2 různé způsoby zbavení se šumu:
\begin{enumerate}
    \item Použití funkce Matlab \texttt{smooth}
    \item Použití NURBS aproximace křivky ze sekce \ref{sec:NURBS aproximace křivky}
\end{enumerate}
Výsledky těchto funkcí jsou na obrázcích č.~\ref{fig: Zpracovaný zašuměný záznam kompenzačního proudu pro dopředný pohyb, Ids = 0 m}, \ref{fig: Zpracovaný zašuměný záznam kompenzačního proudu pro dopředný pohyb, Ids = 1 m}, \ref{fig: Zpracovaný zašuměný záznam kompenzačního proudu pro zpětný pohyb, Ids = 0 m}, \ref{fig: Zpracovaný zašuměný záznam kompenzačního proudu pro zpětný pohyb, Ids = 1 m}. Z takto upravených dat je již lépe vidět, že hodnoty nového kompenzačního proudu lemují příslušnou CCT s výjimkou okolí souřadnice $Prop = 0.5$ a $C_{Arc} = 0.85$. Na tomto místě se hodnoty značně odchylují od původních hodnot kalibrační tabulky, toto je dále lépe vidět na obrázcích č.~\ref{fig: NURBS interpolace zpracovaných dat pomocí funkce smooth pro dopředný pohyb, Ids = 0 m}, \ref{fig: NURBS interpolace zpracovaných dat pomocí funkce smooth pro dopředný pohyb, Ids = 1 m}, \ref{fig: NURBS interpolace zpracovaných dat pomocí funkce smooth pro zpětný pohyb, Ids = 0 m}, \ref{fig: NURBS interpolace zpracovaných dat pomocí funkce smooth pro zpětný pohyb, Ids = 1 m}, a \ref{fig: NURBS interpolace zpracovaných dat pomocí NURBS aproximace pro dopředný pohyb, Ids = 0 m}, \ref{fig: NURBS interpolace zpracovaných dat pomocí NURBS aproximace pro dopředný pohyb, Ids = 1 m}, \ref{fig: NURBS interpolace zpracovaných dat pomocí NURBS aproximace pro zpětný pohyb, Ids = 0 m}, \ref{fig: NURBS interpolace zpracovaných dat pomocí NURBS aproximace pro zpětný pohyb, Ids = 1 m}, kde je uvedena NURBS interpolace zpracovaných dat. Odchylka může být způsobena změnou fyzikálních parametrů manipulátoru a nebo Phillips používá nějaký svůj vlastní algoritmus, který data vyhodnocuje jiným způsobem. Nicméně pro naše účely se tyto odlišnosti hodí, protože nám umožní odzkoušení aktualizace CCT --- viz kapitola~\nameref{section:aktualizace CCT}.
\begin{landscapeimagepage}
    \begin{figure}[H]
        \centering
        \begin{subfigure}{.5\textheight}
            \centering
            \includegraphics[width=0.9\textwidth]{Generated/Proudová odchylka - poloha rychlost - 1.pdf}
            \caption{Trajektorie kloubu $C_{Arc}$ dopředného pohybu pro polohu $Ids = 0$}
            \label{fig: trajektorie kloubu CArc dopředného pohybu Ids=0}
        \end{subfigure}
        \vspace{0.5cm}
        \hspace{2.5cm}
        \begin{subfigure}{.5\textheight}
            \centering
            \includegraphics[width=0.9\textwidth]{Generated/Proudová odchylka - poloha rychlost - 2.pdf}
            \caption{Trajektorie kloubu $C_{Arc}$ dopředného pohybu pro polohu $Ids = 1$}
            \label{fig: trajektorie kloubu CArc dopředného pohybu Ids=1}
        \end{subfigure}
        \vspace{0.5cm}
        \begin{subfigure}{.5\textheight}
            \centering
            \includegraphics[width=0.9\textwidth]{Generated/Proudová odchylka - poloha rychlost - 3.pdf}
            \caption{Trajektorie kloubu $C_{Arc}$ zpětného pohybu pro polohu $Ids = 0$}
            \label{fig: trajektorie kloubu CArc zpětného pohybu Ids=0}
        \end{subfigure}
        \hspace{2.5cm}
        \begin{subfigure}{.5\textheight}
            \centering
            \includegraphics[width=0.9\textwidth]{Generated/Proudová odchylka - poloha rychlost - 4.pdf}
            \caption{Trajektorie kloubu $C_{Arc}$ zpětného pohybu pro polohu $Ids = 1$}
            \label{fig: trajektorie kloubu CArc zpětného pohybu Ids=1}
        \end{subfigure}
        \caption{Ukázka trajektorií kloubu $C_{Arc}$ pro 4 proudové kalibrační tabulky}
        \label{}
    \end{figure}
\end{landscapeimagepage}


\begin{landscapeimagepage}
    \begin{figure}[H]
        \centering
        \begin{subfigure}{.5\textheight}
            \centering
            \includegraphics[width=\textwidth]{Generated/Analýza odchylky proudu - nezpracovaná data - 1.pdf}
            \caption{Vypočtený kompenzační proud ze záznamu testovací trajektorie pro dopředný pohyb, $Ids = 0$}
            \label{fig: Vypočtený kompenzační proud ze záznamu testovací trajektorie pro dopředný pohyb, Ids = 0 m}
        \end{subfigure}
        \vspace{0.5cm}
        \hspace{2.5cm}
        \begin{subfigure}{.5\textheight}
            \centering
            \includegraphics[width=\textwidth]{Generated/Analýza odchylky proudu - nezpracovaná data - 2.pdf}
            \caption{Vypočtený kompenzační proud ze záznamu testovací trajektorie pro dopředný pohyb, $Ids = 1$}
            \label{fig: Vypočtený kompenzační proud ze záznamu testovací trajektorie pro dopředný pohyb, Ids = 1 m}
        \end{subfigure}
        \vspace{0.5cm}
        \begin{subfigure}{.5\textheight}
            \centering
            \includegraphics[width=\textwidth]{Generated/Analýza odchylky proudu - nezpracovaná data - 3.pdf}
            \caption{Vypočtený kompenzační proud ze záznamu testovací trajektorie pro zpětný pohyb, $Ids = 0$}
            \label{fig: Vypočtený kompenzační proud ze záznamu testovací trajektorie pro zpětný pohyb, Ids = 0 m}
        \end{subfigure}
        \hspace{2.5cm}
        \begin{subfigure}{.5\textheight}
            \centering
            \includegraphics[width=\textwidth]{Generated/Analýza odchylky proudu - nezpracovaná data - 4.pdf}
            \caption{Vypočtený kompenzační proud ze záznamu testovací trajektorie pro zpětný pohyb, $Ids = 1$}
            \label{fig: Vypočtený kompenzační proud ze záznamu testovací trajektorie pro zpětný pohyb, Ids = 1 m}
        \end{subfigure}
        \caption{Vypočtený kompenzační proud pro testovací trajektorie}
        \label{}
    \end{figure}
\end{landscapeimagepage}


\begin{landscapeimagepage}
    \begin{figure}[H]
        \centering
        \begin{subfigure}{.5\textheight}
            \centering
            \includegraphics[width=\textwidth]{Generated/Analýza odchylky proudu - zpracovaná data - 1.pdf}
            \caption{Zpracovaný zašuměný záznam kompenzačního proudu pro dopředný pohyb, $Ids = 0$}
            \label{fig: Zpracovaný zašuměný záznam kompenzačního proudu pro dopředný pohyb, Ids = 0 m}
        \end{subfigure}
        \vspace{0.5cm}
        \hspace{2.5cm}
        \begin{subfigure}{.5\textheight}
            \centering
            \includegraphics[width=\textwidth]{Generated/Analýza odchylky proudu - zpracovaná data - 2.pdf}
            \caption{Zpracovaný zašuměný záznam kompenzačního proudu pro dopředný pohyb, $Ids = 1$}
            \label{fig: Zpracovaný zašuměný záznam kompenzačního proudu pro dopředný pohyb, Ids = 1 m}
        \end{subfigure}
        \vspace{0.5cm}
        \begin{subfigure}{.5\textheight}
            \centering
            \includegraphics[width=\textwidth]{Generated/Analýza odchylky proudu - zpracovaná data - 3.pdf}
            \caption{Zpracovaný zašuměný záznam kompenzačního proudu pro zpětný pohyb, $Ids = 0$}
            \label{fig: Zpracovaný zašuměný záznam kompenzačního proudu pro zpětný pohyb, Ids = 0 m}
        \end{subfigure}
        \hspace{2.5cm}
        \begin{subfigure}{.5\textheight}
            \centering
            \includegraphics[width=\textwidth]{Generated/Analýza odchylky proudu - zpracovaná data - 4.pdf}
            \caption{Zpracovaný zašuměný záznam kompenzačního proudu pro zpětný pohyb, $Ids = 1$}
            \label{fig: Zpracovaný zašuměný záznam kompenzačního proudu pro zpětný pohyb, Ids = 1 m}
        \end{subfigure}
        \caption{Zpracování zašuměných záznamů kompenzačního proudu pro testovací trajektorie}
        \label{}
    \end{figure}
\end{landscapeimagepage}


\begin{landscapeimagepage}
    \begin{figure}[H]
        \centering
        \begin{subfigure}{.5\textheight}
            \centering
            \includegraphics[width=\textwidth]{Generated/Interpolace CCT - smooth - 1.pdf}
            \caption{NURBS interpolace zpracovaných dat pomocí funkce \texttt{smooth} pro dopředný pohyb, $Ids = 0$}
            \label{fig: NURBS interpolace zpracovaných dat pomocí funkce smooth pro dopředný pohyb, Ids = 0 m}
        \end{subfigure}
        \vspace{0.5cm}
        \hspace{2.5cm}
        \begin{subfigure}{.5\textheight}
            \centering
            \includegraphics[width=\textwidth]{Generated/Interpolace CCT - smooth - 2.pdf}
            \caption{NURBS interpolace zpracovaných dat pomocí funkce \texttt{smooth} pro dopředný pohyb, $Ids = 1$}
            \label{fig: NURBS interpolace zpracovaných dat pomocí funkce smooth pro dopředný pohyb, Ids = 1 m}
        \end{subfigure}
        \vspace{0.5cm}
        \begin{subfigure}{.5\textheight}
            \centering
            \includegraphics[width=\textwidth]{Generated/Interpolace CCT - smooth - 3.pdf}
            \caption{NURBS interpolace zpracovaných dat pomocí funkce \texttt{smooth} pro zpětný pohyb, $Ids = 0$}
            \label{fig: NURBS interpolace zpracovaných dat pomocí funkce smooth pro zpětný pohyb, Ids = 0 m}
        \end{subfigure}
        \hspace{2.5cm}
        \begin{subfigure}{.5\textheight}
            \centering
            \includegraphics[width=\textwidth]{Generated/Interpolace CCT - smooth - 4.pdf}
            \caption{NURBS interpolace zpracovaných dat pomocí funkce \texttt{smooth} pro zpětný pohyb, $Ids = 1$}
            \label{fig: NURBS interpolace zpracovaných dat pomocí funkce smooth pro zpětný pohyb, Ids = 1 m}
        \end{subfigure}
        \caption{Proložení zpracovaných dat pomocí \texttt{smooth} NURBS interpolací}
        \label{}
    \end{figure}
\end{landscapeimagepage}

\begin{landscapeimagepage}
    \begin{figure}[H]
        \centering
        \begin{subfigure}{.5\textheight}
            \centering
            \includegraphics[width=\textwidth]{Generated/Interpolace CCT - nurbs approx - 1.pdf}
            \caption{NURBS interpolace zpracovaných dat pomocí NURBS aproximace pro dopředný pohyb, $Ids = 0$}
            \label{fig: NURBS interpolace zpracovaných dat pomocí NURBS aproximace pro dopředný pohyb, Ids = 0 m}
        \end{subfigure}
        \vspace{0.5cm}
        \hspace{2.5cm}
        \begin{subfigure}{.5\textheight}
            \centering
            \includegraphics[width=\textwidth]{Generated/Interpolace CCT - nurbs approx - 2.pdf}
            \caption{NURBS interpolace zpracovaných dat pomocí NURBS aproximace pro dopředný pohyb, $Ids = 1$}
            \label{fig: NURBS interpolace zpracovaných dat pomocí NURBS aproximace pro dopředný pohyb, Ids = 1 m}
        \end{subfigure}
        \vspace{0.5cm}
        \begin{subfigure}{.5\textheight}
            \centering
            \includegraphics[width=\textwidth]{Generated/Interpolace CCT - nurbs approx - 3.pdf}
            \caption{NURBS interpolace zpracovaných dat pomocí NURBS aproximace pro zpětný pohyb, $Ids = 0$}
            \label{fig: NURBS interpolace zpracovaných dat pomocí NURBS aproximace pro zpětný pohyb, Ids = 0 m}
        \end{subfigure}
        \hspace{2.5cm}
        \begin{subfigure}{.5\textheight}
            \centering
            \includegraphics[width=\textwidth]{Generated/Interpolace CCT - nurbs approx - 4.pdf}
            \caption{NURBS interpolace zpracovaných dat pomocí NURBS aproximace pro zpětný pohyb, $Ids = 1$}
            \label{fig: NURBS interpolace zpracovaných dat pomocí NURBS aproximace pro zpětný pohyb, Ids = 1 m}
        \end{subfigure}
        \caption{Proložení zpracovaných dat pomocí NURBS aproximace NURBS interpolací}
        \label{}
    \end{figure}
\end{landscapeimagepage}