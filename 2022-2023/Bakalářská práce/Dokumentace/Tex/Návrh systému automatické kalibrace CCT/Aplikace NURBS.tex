\section{Návrh systému automatické kalibrace proudové tabulky}\label{section: návrh automatické kalibraci}
Tato kapitola se již zabývá aplikací NURBS teorie na náš problém --- tj.~užití NURBS teorie pro práci s~kalibrační tabulkou. Dále se kapitola zaměřuje na sběr vhodných hodnot pro aktualizaci kalibrační tabulky.
\subsection{Aplikace NURBS teorie pro modelování kalibrační tabulky}
V této kapitole je již použita NURBS interpolace a~aproximace na konkrétních datech. Nejprve je uvedena ukázka na 1 DoF modelu, poté na řezu kalibrační tabulkou, kde lze vidět cílenou aproximaci, dále je uvedena 3D a~4D interpolace kompenzačních tabulek.
\subsubsection{1 DoF model}
Na obrázku č.~\ref{fig:Cílená aproximace křivky 1 DoF} je ukázka cílené aproximace křivky pro náš 1 DoF model systému. Aproximační body jsme získali úpravou fyzikálních parametrů modelu, zvolené body jsou záměrně mimo "mřížku" od původních bodů. Ukázka obsahuje výsledky pro dvě různé váhy aproximačních bodů.
\subsubsection{2D řez kalibrační tabulkou}
Na obrázcích č.~\ref{fig:Cílená aproximace křivky}, \ref{fig:Cílená aproximace křivky přiblíženo} je ukázka
cílené aproximace křivky na jedné z~kalibračních tabulek pro fixní hodnotu osy $Prop$, kde je lépe ukázáno jak cílená aproximace funguje v~detailu. Řez tabulkou jsme získali právě zvolením fixní hodnoty kloubu $Prop$.
\subsubsection{Interpolace kalibračních tabulek}
Na obrázcích č.~\ref{fig: Ukázka interpolace na reálných datech č. 1}, \ref{fig: Ukázka interpolace na reálných datech č. 2}, \ref{fig: Ukázka interpolace na reálných datech č. 3}, \ref{fig: Ukázka interpolace na reálných datech č. 4} lze napravo vidět aplikaci NURBS interpolace povrchu a~nalevo lze vidět původní data.
Stejně jako v předchozích ukázkách se i zde používají bázové funkce s polynomy 3. stupně, tj. $p = q = 3$.
\subsubsection{Interpolace napříč kalibračními tabulkami}
Jak již bylo zmíněno, máme 2 kalibrační tabulky pro dvě polohy kloubu $Ids$, a~proto bychom rádi chtěli přes tuto souřadnici interpolovat, abychom získali kalibrační tabulky mezi těmito dvěma polohami. K~tomuto využijeme NURBS interpolaci nadpovrchu ze sekce \nameref{section:interpolace 4D povrchu}, výsledné animace jsou na obrázcích č.~\ref{fig:Demo 4D Interpolace mezi povrchy - reálná data dopředný pohyb},
\ref{fig:Demo 4D Interpolace mezi povrchy - reálná data zpětný pohyb}.
\begin{figure}[H]
    \centering
    \includegraphics[width=0.75\textwidth]{Generated/Ukázka cílené aproximace na 1DoF modelu.pdf}
    \caption{Cílená aproximace křivky metodou nejmenších čtverců na 1 DoF modelu
        pro různé váhy nových bodů}
    \label{fig:Cílená aproximace křivky 1 DoF}
\end{figure}
\begin{imagepage}
    \begin{figure}[H]
        \centering
        \includegraphics[width=0.85\textwidth]{Generated/Ukázka cílené aproximace na reálných datech.pdf}
        \caption{Cílená aproximace křivky metodou nejmenších čtverců na reálných datech}
        \label{fig:Cílená aproximace křivky}
    \end{figure}
    \begin{figure}[H]
        \centering
        \includegraphics[width=0.85\textwidth]{Generated/Ukázka cílené aproximace na reálných datech přiblíženo.pdf}
        \caption{Cílená aproximace křivky metodou nejmenších čtverců na reálných datech (přiblíženo)}
        \label{fig:Cílená aproximace křivky přiblíženo}
    \end{figure}
\end{imagepage}

\begin{landscapeimagepage}
    % \vspace*{\fill}
    \begin{figure}[H]
        \centering
        \begin{subfigure}{.5\textheight}
            \centering
            \includegraphics[width=\textwidth]{Generated/Plot kompenzace proudu - reálná data 1.pdf}
            \caption{Lineární interpolace kompenzační tabulky proudu č. 1}
            \label{fig:Lineární interpolace kompenzační tabulky proudu č. 1}
        \end{subfigure}
        \hspace{2.5cm}
        \begin{subfigure}{.5\textheight}
            \centering
            \includegraphics[width=\textwidth]{Generated/Plot kompenzace proudu - reálná data - interpolace1.pdf}
            \caption{NURBS interpolace kompenzační tabulky proudu č. 1}
            \label{fig:NURBS interpolace kompenzační tabulky proudu č. 1}
        \end{subfigure}
        \caption{Ukázka interpolace na reálných datech č. 1}
        \label{fig: Ukázka interpolace na reálných datech č. 1}
    \end{figure}
    \begin{figure}[H]
        \centering
        \begin{subfigure}{.5\textheight}
            \centering
            \includegraphics[width=\textwidth]{Generated/Plot kompenzace proudu - reálná data 2.pdf}
            \caption{Lineární interpolace kompenzační tabulky proudu č. 2}
            \label{fig:Lineární interpolace kompenzační tabulky proudu č. 2}
        \end{subfigure}
        \hspace{2.5cm}
        \begin{subfigure}{.5\textheight}
            \centering
            \includegraphics[width=\textwidth]{Generated/Plot kompenzace proudu - reálná data - interpolace2.pdf}
            \caption{NURBS interpolace kompenzační tabulky proudu č. 2}
            \label{fig:NURBS interpolace kompenzační tabulky proudu č. 2}
        \end{subfigure}
        \caption{Ukázka interpolace na reálných datech č. 2}
        \label{fig: Ukázka interpolace na reálných datech č. 2}
    \end{figure}
    % \vspace*{\fill}
\end{landscapeimagepage}

\begin{landscapeimagepage}
    % \vspace*{\fill}
    \begin{figure}[H]
        \centering
        \begin{subfigure}{.5\textheight}
            \centering
            \includegraphics[width=\textwidth]{Generated/Plot kompenzace proudu - reálná data 3.pdf}
            \caption{Lineární interpolace kompenzační tabulky proudu č. 3}
            \label{fig:Lineární interpolace kompenzační tabulky proudu č. 3}
        \end{subfigure}
        \hspace{2.5cm}
        \begin{subfigure}{.5\textheight}
            \centering
            \includegraphics[width=\textwidth]{Generated/Plot kompenzace proudu - reálná data - interpolace3.pdf}
            \caption{NURBS interpolace kompenzační tabulky proudu č. 3}
            \label{fig:NURBS interpolace kompenzační tabulky proudu č. 3}
        \end{subfigure}
        \caption{Ukázka interpolace na reálných datech č. 3}
        \label{fig: Ukázka interpolace na reálných datech č. 3}
    \end{figure}
    \begin{figure}[H]
        \centering
        \begin{subfigure}{.5\textheight}
            \centering
            \includegraphics[width=\textwidth]{Generated/Plot kompenzace proudu - reálná data 4.pdf}
            \caption{Lineární interpolace kompenzační tabulky proudu č. 4}
            \label{fig:Lineární interpolace kompenzační tabulky proudu č. 4}
        \end{subfigure}
        \hspace{2.5cm}
        \begin{subfigure}{.5\textheight}
            \centering
            \includegraphics[width=\textwidth]{Generated/Plot kompenzace proudu - reálná data - interpolace4.pdf}
            \caption{NURBS interpolace kompenzační tabulky proudu č. 4}
            \label{fig:NURBS interpolace kompenzační tabulky proudu č. 4}
        \end{subfigure}
        \caption{Ukázka interpolace na reálných datech č. 4}
        \label{fig: Ukázka interpolace na reálných datech č. 4}
    \end{figure}
    % \vspace*{\fill}
\end{landscapeimagepage}
\ifthenelse{\boolean{includeAnimations}}{
    \begin{landscapeimagepage}
        % \vspace*{\fill}
        \begin{figure}[H]
            \centering
            \includegraphics[width=0.45\pdfpagewidth]{Generated/Interpolace 4d povrchu část 1 č. 2.pdf}\hspace{0.5cm}
            \includegraphics[width=0.45\pdfpagewidth]{Generated/Interpolace 4d povrchu část 2 č. 2.pdf}
            \caption{Interpolační povrchy pro 4D interpolaci kompenzačních tabulek - dopředný pohyb}
        \end{figure}
        \begin{figure}[H]
            \centering
            \includeanimation{Generated/4D surface demo 2/frame-}{12.5}{palindrome, height=0.4\textheight}
            % \includeanimation{Generated/4D surface demo 1/frame-}{12.5}{palindrome, height=0.4\textheight}
            \caption{Ukázka průběhu interpolace mezi kompenzačními tabulkami přes souřadnici $Ids$ --- dopředný pohyb}
            \label{fig:Demo 4D Interpolace mezi povrchy - reálná data dopředný pohyb}
        \end{figure}
        % \vspace*{\fill}
    \end{landscapeimagepage}}{}
\ifthenelse{\boolean{includeAnimationFrames}}{
    \begin{landscapeimagepage}
        % \vspace*{\fill}
        \begin{figure}[H]
            \centering
            \includegraphics[width=0.33\pdfpagewidth]{Generated/Interpolace 4d povrchu část 1 č. 2.pdf}\hspace{0.5cm}
            \includegraphics[width=0.33\pdfpagewidth]{Generated/Interpolace 4d povrchu část 2 č. 2.pdf}
            \caption{Interpolační povrchy pro 4D interpolaci kompenzačních tabulek - dopředný pohyb}
        \end{figure}
        \begin{figure}[H]
            \centering
            \includeanimationframes{Generated/4D surface demo 2/frame-}{height=0.25\pdfpagewidth}
            % \includeanimation{Generated/4D surface demo 1/frame-}{12.5}{palindrome, height=0.4\textheight}
            \caption{Ukázka průběhu interpolace mezi kompenzačními tabulkami přes souřadnici $Ids$ --- dopředný pohyb}
            \label{fig:Demo 4D Interpolace mezi povrchy - reálná data dopředný pohyb}
        \end{figure}
        % \vspace*{\fill}
    \end{landscapeimagepage}}{}

\ifthenelse{\boolean{includeAnimations}}{
    \begin{landscapeimagepage}
        % \vspace*{\fill}
        \begin{figure}[H]
            \centering
            \includegraphics[width=0.45\pdfpagewidth]{Generated/Interpolace 4d povrchu část 2 č. 3.pdf}\hspace{0.5cm}
            \includegraphics[width=0.45\pdfpagewidth]{Generated/Interpolace 4d povrchu část 1 č. 3.pdf}
            \caption{Interpolační povrchy pro 4D interpolaci kompenzačních tabulek - zpětný pohyb}
        \end{figure}
        \vfill
        \begin{figure}[H]
            \centering
            \includeanimation{Generated/4D surface demo 3/frame-}{12.5}{palindrome, height=0.4\textheight}
            % \includeanimation{Generated/4D surface demo 1/frame-}{12.5}{palindrome, height=0.5\textheight}
            \caption{Ukázka průběhu interpolace mezi kompenzačními tabulkami přes souřadnici $Ids$ --- zpětný pohyb}
            \label{fig:Demo 4D Interpolace mezi povrchy - reálná data zpětný pohyb}
        \end{figure}
        % \vspace*{\fill}
    \end{landscapeimagepage}}{}
\ifthenelse{\boolean{includeAnimationFrames}}{
    \begin{landscapeimagepage}
        % \vspace*{\fill}
        \begin{figure}[H]
            \centering
            \includegraphics[width=0.33\pdfpagewidth]{Generated/Interpolace 4d povrchu část 2 č. 3.pdf}\hspace{0.5cm}
            \includegraphics[width=0.33\pdfpagewidth]{Generated/Interpolace 4d povrchu část 1 č. 3.pdf}
            \caption{Interpolační povrchy pro 4D interpolaci kompenzačních tabulek - zpětný pohyb}
        \end{figure}
        \vfill
        \begin{figure}[H]
            \centering
            \includeanimationframes{Generated/4D surface demo 3/frame-}{height=0.25\pdfpagewidth}
            % \includeanimation{Generated/4D surface demo 1/frame-}{12.5}{palindrome, height=0.5\textheight}
            \caption{Ukázka průběhu interpolace mezi kompenzačními tabulkami přes souřadnici $Ids$ --- zpětný pohyb}
            \label{fig:Demo 4D Interpolace mezi povrchy - reálná data zpětný pohyb}
        \end{figure}
        % \vspace*{\fill}
    \end{landscapeimagepage}}{}
