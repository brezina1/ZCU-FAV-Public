\subsection[NURBS povrchy]{NURBS povrchy\footcite[kapitola 3.4]{The_NURBS_Book}}
Pro sestrojení NURBS povrchu potřebujeme obousměrnou síť řídících bodů $\bm{P}_{i,j}$ a
dva uzlové vektory $\bm{U}$ a $\bm{V}$, poté je možné sestrojit povrch $\bm{S}(u,v)$\footnote{Jedná se tedy o~vektorovou funkci dvou skalárních proměnných.}:
\begin{align}
    \bm{S}(u,v) = \sum_{i=0}^{n}\sum_{j=0}^{m}N_{i,p}(u)N_{j,q}(v)\bm{P}_{i,j}
\end{align}
kde
\begin{itemize}
    \item $p$ značí stupeň křivek ve směru $u$
    \item $q$ značí stupeň křivek ve směru $v$
    \item $n$ značí počet řídících bodů ve směru $u$
    \item $m$ značí počet řídících bodů ve směru $v$
    \item $\bm{P}_{i,j}$ je síť řídících bodů, $\dim{\bm{P}_{i,j}} \ge 3$
    \item $u \in \bm{U}$, $v \in \bm{V}$
    \item $a \le u \le b$, $a \le v \le b$ --- stejně jako u \nameref{section: NURBS křivky} budeme uvažovat $a = 0$, $b = 1$ bez ztráty obecnosti
\end{itemize}
Pro uzlové vektory $\bm{U}$ a $\bm{V}$ platí:
\begin{align}
    \bm{U} = \{\overbrace{\underbrace{a, \ldots, a}_{p + 1}, u_{p + 1}, \ldots, u_{r -p - 1}, \underbrace{b, \ldots, b}_{p + 1}}^{r + 1}\}\\
    \bm{V} = \{\overbrace{\underbrace{a, \ldots, a}_{q + 1}, u_{q + 1}, \ldots, u_{s -q - 1}, \underbrace{b, \ldots, b}_{q + 1}}^{s + 1}\}
\end{align}
kde $r = n + p + 1$, $s = m + q + 1$. \par Definice bázových funkcí zůstává stejná --- viz (\ref{eq:bázová funkce}).