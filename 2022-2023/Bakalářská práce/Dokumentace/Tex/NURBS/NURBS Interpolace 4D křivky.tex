\subsection{Interpolace 4D křivky}
Interpolací 4D křivky je v~tomto případě myšlena křivka tvořena 3D body ($x, y,
    z$), které se mění v~závislosti na 4. souřadnici $w$. Tudíž nechceme
interpolovat jednu dlouhou 4D křivku (jak jsme dosud dělali v~předchozích
případech), ale vlastně několik 3D křivek mezi sebou za využití 4.
souřadnice.\par Tohoto docílíme interpolací samotných bodů příslušných křivek
přes souřadnici $w$, kterou lze vidět na obrázcích č. \ref{fig:Demo 4D
    Interpolace mezi body přes souřadnici w 1}, \ref{fig:Demo 4D Interpolace mezi
    body přes souřadnici w 2}, \ref{fig:Demo 4D Interpolace mezi body přes
    souřadnici w 3}. Dále obrázky č. \ref{fig:Demo 4D Interpolace mezi body přes
    souřadnici w shora 1}, \ref{fig:Demo 4D Interpolace mezi body přes souřadnici w
    shora 2} a \ref{fig:Demo 4D Interpolace mezi body přes souřadnici w shora 3}
značí pohled na přechodové křivky shora, a~obrázky č. \ref{fig:Demo 4D
    Interpolace mezi body přes souřadnici w konst x 1}, \ref{fig:Demo 4D
    Interpolace mezi body přes souřadnici w konst x 2} a \ref{fig:Demo 4D
    Interpolace mezi body přes souřadnici w konst x 3} ukazují interpolaci mezi
jedním bodem z~každé křivky.

\subsubsection{4D Křivka číslo 1}
Tuto 4D křivku tvoří tři 3D křivky popsané body:
\begin{alignat}{3}
    \bm{Q}_1 & = [\bm{x}, \bm{y}, \bm{z}, \bm{w}] & = & [\bm{x}, \sin(\bm{x}), 10, 0]   \\
    \bm{Q}_2 & = [\bm{x}, \bm{y}, \bm{z}, \bm{w}] & = & [\bm{x}, -\sin(\bm{x}), 0, 100] \\
    \bm{Q}_3 & = [\bm{x}, \bm{y}, \bm{z}, \bm{w}] & = & [\bm{x}, \sin(\bm{x}), 20, 200]
\end{alignat}
kde
\begin{equation}
    \bm{x} = linspace(0, 2\cdot\pi, 25)
\end{equation}
Tyto fáze naší 4D křivky jsou rovnoměrně rozmístěny ve 4. dimenzi, tj.
$\bm{w}$ = $0, 100, 200$ při křivky 1, 2 a~3 respektive. Tímto by přechod mezi křivkami měl
být \todo{rovnoměrný? plynulý? symetrický?}. Ukázka přechodu mezi těmito křivkami je
zobrazena na animaci č. \ref{fig:Demo 4D Interpolace mezi křivkami č. 1}.
\subsubsection{4D Křivka číslo 2}
Tuto 4D křivku tvoří tři 3D křivky popsané body:
\begin{alignat}{3}
    \bm{Q}_1 & = [\bm{x}, \bm{y}, \bm{z}, \bm{w}] & = & [\bm{x}, \sin(\bm{x}), 10, 0]                            \\
    \bm{Q}_2 & = [\bm{x}, \bm{y}, \bm{z}, \bm{w}] & = & [\bm{x}, -\sin(\bm{x}), 0, \text{\textcolor{red}{$50$}}] \\
    \bm{Q}_3 & = [\bm{x}, \bm{y}, \bm{z}, \bm{w}] & = & [\bm{x}, \sin(\bm{x}), 20, 200]
\end{alignat}
kde
\begin{equation}
    \bm{x} = linspace(0, 2\cdot\pi, 25)
\end{equation}
V tomto případě interpolační křivky již nejsou rozmístěny rovnoměrně ve 4. dimenzi,
což v~našem případě znamená, že přechod od křivky tvořenou body $\bm{Q}_1$ do křivky
tvořenou $\bm{Q}_2$ je $4\times$ kratší, než přechod od $\bm{Q}_2$ do $\bm{Q}_3$. Toto
je naznačené na obrázcích č.~\ref{fig:Demo 4D Interpolace mezi body přes souřadnici w 2},
\ref{fig:Demo 4D Interpolace mezi body přes souřadnici w shora 2} a
\ref{fig:Demo 4D Interpolace mezi body přes souřadnici w konst x 2},
kde je nepoměr vidět (přestože je zkreslený).
Pořádně lze tento jev vidět až na animaci č. \ref{fig:Demo 4D Interpolace mezi křivkami č. 2},
kde je také celý průběh interpolace.

\subsubsection{4D Křivka číslo 3}
Tuto 4D křivku tvoří tři 3D křivky popsané body:
\begin{alignat}{3}
    \bm{Q}_1 & = [\bm{x}, \bm{y}, \bm{z}, \bm{w}] & = & [\bm{x}, \sin(\bm{x}), 10, 0]                        \\
    \bm{Q}_2 & = [\bm{x}, \bm{y}, \bm{z}, \bm{w}] & = & [\bm{x}, -\sin(\bm{x}), 2.5 + 2.5\cos(2\bm{x}), 100] \\
    \bm{Q}_3 & = [\bm{x}, \bm{y}, \bm{z}, \bm{w}] & = & [\bm{x}, \sin(\bm{x}), 17.5 - 2.5\cos(2\bm{x}), 200]
\end{alignat}
kde
\begin{equation}
    \bm{x} = linspace(0, 2\cdot\pi, 25)
\end{equation}
Zde se pro ukázku interpolované křivky mění i~v~ose $z$ v~závislosti na $x$.
Obrázky k~této interpolaci jsou \ref{fig:Demo 4D Interpolace mezi body přes souřadnici w 3},
\ref{fig:Demo 4D Interpolace mezi body přes souřadnici w shora 3},
\ref{fig:Demo 4D Interpolace mezi body přes souřadnici w konst x 3} a
\ref{fig:Demo 4D Interpolace mezi křivkami č. 3}.

\begin{imagepage}
    \begin{figure}[H]
        \centering
        \includegraphics[height=0.4\textheight]{Generated/Interpolace 4D křivky - Interpolace mezi body ve 4. dimenzi č. 1.pdf}
        \caption{Interpolace bodů křivek přes souřadnici $w$ č. 1}
        \label{fig:Demo 4D Interpolace mezi body přes souřadnici w 1}
    \end{figure}
    \begin{figure}[H]
        \centering
        \includegraphics[height=0.4\textheight]{Generated/Interpolace 4D křivky - Interpolace mezi body ve 4. dimenzi (shora) č. 1.pdf}
        \caption{Interpolace bodů křivek přes souřadnici $w$ (pohled shora) č. 1}
        \label{fig:Demo 4D Interpolace mezi body přes souřadnici w shora 1}
    \end{figure}
\end{imagepage}

\begin{imagepage}
    \begin{figure}[H]
        \centering
        \includegraphics[height=0.4\textheight]{Generated/Interpolace 4D křivky - Interpolace mezi body ve 4. dimenzi 1 vlákno č. 1.pdf}
        \caption{Ukázka interpolace pro 1 bod z~každé křivky č. 1}
        \label{fig:Demo 4D Interpolace mezi body přes souřadnici w konst x 1}
    \end{figure}
    \ifthenelse{\boolean{includeAnimations}}{
        \begin{figure}[H]
            \centering
            \includeanimation{Generated/4D curve demo 1/frame-}{12.5}{palindrome, height=0.4\textheight}
            \caption{Ukázka průběhu interpolace mezi křivkami přes souřadnici $w$ č. 1}
            \label{fig:Demo 4D Interpolace mezi křivkami č. 1}
        \end{figure}
    }{}
\end{imagepage}
\ifthenelse{\boolean{includeAnimationFrames}}{
    \begin{landscapeimagepage}
        \vspace*{\fill}
        \begin{figure}[H]
            \centering
            \includeanimationframes[\vspace{1cm}]{Generated/4D curve demo 1/frame-}{width=0.28\pdfpageheight}
            \caption{Ukázka průběhu interpolace mezi křivkami přes souřadnici $w$ č. 1}
            \label{fig:Demo 4D Interpolace mezi křivkami č. 1}
        \end{figure}
        \vspace*{\fill}
    \end{landscapeimagepage}}{}
% ======================================
\begin{imagepage}
    \begin{figure}[H]
        \centering
        \includegraphics[height=0.4\textheight]{Generated/Interpolace 4D křivky - Interpolace mezi body ve 4. dimenzi č. 2.pdf}
        \caption{Interpolace bodů křivek přes souřadnici $w$ č. 2}
        \label{fig:Demo 4D Interpolace mezi body přes souřadnici w 2}
    \end{figure}
    \begin{figure}[H]
        \centering
        \includegraphics[height=0.4\textheight]{Generated/Interpolace 4D křivky - Interpolace mezi body ve 4. dimenzi (shora) č. 2.pdf}
        \caption{Interpolace bodů křivek přes souřadnici $w$ (pohled shora) č. 2}
        \label{fig:Demo 4D Interpolace mezi body přes souřadnici w shora 2}
    \end{figure}
\end{imagepage}

\begin{imagepage}
    \begin{figure}[H]
        \centering
        \includegraphics[height=0.4\textheight]{Generated/Interpolace 4D křivky - Interpolace mezi body ve 4. dimenzi 1 vlákno č. 2.pdf}
        \caption{Ukázka interpolace pro 1 bod z~každé křivky č. 2}
        \label{fig:Demo 4D Interpolace mezi body přes souřadnici w konst x 2}
    \end{figure}
    \ifthenelse{\boolean{includeAnimations}}{
        \begin{figure}[H]
            \centering
            \ifthenelse{\boolean{includeAnimations}}{\includeanimation{Generated/4D curve demo 2/frame-}{12.5}{palindrome, height=0.4\textheight}}{}
            \caption{Ukázka průběhu interpolace mezi křivkami přes souřadnici $w$ č. 2}
            \label{fig:Demo 4D Interpolace mezi křivkami č. 2}
        \end{figure}
    }{}
\end{imagepage}
\ifthenelse{\boolean{includeAnimationFrames}}{
    \begin{landscapeimagepage}
        \vspace*{\fill}
        \begin{figure}[H]
            \centering
            \includeanimationframes[\vspace{1cm}]{Generated/4D curve demo 2/frame-}{width=0.28\pdfpageheight}
            \caption{Ukázka průběhu interpolace mezi křivkami přes souřadnici $w$ č. 2}
            \label{fig:Demo 4D Interpolace mezi křivkami č. 2}
        \end{figure}
        \vspace*{\fill}
    \end{landscapeimagepage}}{}
% =====================================
\begin{imagepage}
    \begin{figure}[H]
        \centering
        \includegraphics[height=0.4\textheight]{Generated/Interpolace 4D křivky - Interpolace mezi body ve 4. dimenzi č. 3.pdf}
        \caption{Interpolace bodů křivek přes souřadnici $w$ č. 3 \todo{možná jinej pohled}}
        \label{fig:Demo 4D Interpolace mezi body přes souřadnici w 3}
    \end{figure}
    \begin{figure}[H]
        \centering
        \includegraphics[height=0.4\textheight]{Generated/Interpolace 4D křivky - Interpolace mezi body ve 4. dimenzi (shora) č. 3.pdf}
        \caption{Interpolace bodů křivek přes souřadnici $w$ (pohled shora) č. 3}
        \label{fig:Demo 4D Interpolace mezi body přes souřadnici w shora 3}
    \end{figure}
\end{imagepage}

\begin{imagepage}
    \begin{figure}[H]
        \centering
        \includegraphics[height=0.4\textheight]{Generated/Interpolace 4D křivky - Interpolace mezi body ve 4. dimenzi 1 vlákno č. 3.pdf}
        \caption{Ukázka interpolace pro 1 bod z~každé křivky č. 3}
        \label{fig:Demo 4D Interpolace mezi body přes souřadnici w konst x 3}
    \end{figure}
    \ifthenelse{\boolean{includeAnimations}}{
        \begin{figure}[H]
            \centering
            \ifthenelse{\boolean{includeAnimations}}{\includeanimation{Generated/4D curve demo 3/frame-}{12.5}{palindrome, height=0.4\textheight}}{}
            \caption{Ukázka průběhu interpolace mezi křivkami přes souřadnici $w$ č. 3 \todo{možná jinej pohled}}
            \label{fig:Demo 4D Interpolace mezi křivkami č. 3}
        \end{figure}
    }{}
\end{imagepage}
\ifthenelse{\boolean{includeAnimationFrames}}{
    \begin{landscapeimagepage}
        \vspace*{\fill}
        \begin{figure}[H]
            \centering
            \includeanimationframes[\vspace{1cm}]{Generated/4D curve demo 3/frame-}{width=0.28\pdfpageheight}
            \caption{Ukázka průběhu interpolace mezi křivkami přes souřadnici $w$ č. 3}
            \label{fig:Demo 4D Interpolace mezi křivkami č. 3}
        \end{figure}
        \vspace*{\fill}
    \end{landscapeimagepage}}{}