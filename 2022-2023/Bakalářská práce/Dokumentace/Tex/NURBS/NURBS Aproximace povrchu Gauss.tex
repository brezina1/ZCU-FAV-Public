\subsubsection{NURBS aproximace povrchu s užitím Gaussovy funkce}\label{section: gauss surface approximation 3D}
Metoda v sekci \nameref{section:aproximace povrchu} vyžaduje, aby aproximační
body ležely v mřížce (3D i 4D případ), a proto není vhodná pro účely této
práce. Proto jsem přišel s následující metodou, která sice není postavena na
žádném matematickém kritériu (jako například nejmenší čtverce v předchozích
aproximacích), ale nevyžaduje, aby aproximační body ležely v mřížce, a poskytuje
postačující výsledky. \par Metoda využívá váženého průměrování a Gaussovy
funkce $f(x, y)$:
\begin{align}
    f(x,y) = \exp\left(-\left(\frac{(x - x_0)^2}{2\sigma_x^2} + \frac{(y - y_0)^2}{2\sigma_y^2}\right)\right)
\end{align}
kde
\begin{itemize}
    % \item $w_i$ - váhu bodu $\bm{Q}_i$
    \item $x_0$ -- střední hodnota osy $x$
    \item $y_0$ -- střední hodnota osy $y$
    \item $\sigma_x$ -- variance osy $x$
    \item $\sigma_y$ -- variance osy $y$
\end{itemize}
Užitím Gaussovy funkce je zajištěno, že váha aproximovaného bodu se vzdáleností klesá k nule
(viz obrázky č.~\ref{fig:Ukázka Gaussovo funkce 3D}~a~\ref{fig:Ukázka Gaussovo funkce 4D}),
díky tomu je možné provádět vážený průměr přes všechny body\footnote{V případě nadměrného množství bodů lze průměrování omezit na okolí aproximovaného bodu pro snížení výpočetní náročnosti.}.\par
Algoritmus vyžaduje libovolnou množinu původních bodů $\bm{Q}_i$, množinu
aproximačních bodů $\bar{\bm{Q}}_j$ s množinou příslušných vah $\omega_j$ a již
zmíněné parametry $\sigma_x$, $\sigma_y$\footnote{V případě potřeby algoritmus lze snadno
    upravit tak, že parametry $\sigma_x$ a $\sigma_y$ mohou být zadány individuálně
    pro každý aproximační bod $\bar{\bm{Q}}_j$}. Kroky algoritmu jsou následující:
\begin{enumerate}
    \item \label{gauss approx krok 1} Zvol další aproximační bod $\bar{\bm{Q}} = [x_{ap}, y_{ap}, z_{ap}]$ a jeho váhu $w$
    \item Projdi všechny body $\bm{Q}_i = [x,y,z]$ a pro každý proveď:
          \begin{enumerate}
              \item $\bar{\omega} = \omega \cdot f(x, y) \quad \quad x_0 = x_{ap},~y_0 = y_{ap}$\\
                    {\tiny{(Výpočet upravené váhy na základě vzdálenosti od aproximovaného bodu)}}
              \item $\bar{z} = \frac{z + z_{ap} \cdot \bar{\omega} }{\bar{\omega} + 1}$\\
                    {\tiny{(Výpočet nové hodnoty $z$ na základě váženého průměru)}}
              \item Nahraď $z$ za $\bar{z}$
          \end{enumerate}
    \item Vrať se na krok~\ref{gauss approx krok 1}, pokud si neprošel všechny
          aproximační body, jinak konec
\end{enumerate}
Pomocí parametrů $\sigma_x, \sigma_y$ lze upřednostnit aproximaci v ose $x$ nebo $y$. Tento efekt lze vidět na obrázcích
č.~\ref{fig:Ukázka aproximace povrchu s užitím Gaussovo funkce č. 1},
\ref{fig:Ukázka aproximace povrchu s užitím Gaussovo funkce č. 2},
\ref{fig:Ukázka aproximace povrchu s užitím Gaussovo funkce č. 3},
\ref{fig:Ukázka aproximace povrchu s užitím Gaussovo funkce č. 4},
\ref{fig:Ukázka aproximace povrchu s užitím Gaussovo funkce č. 5},
\ref{fig:Ukázka aproximace povrchu s užitím Gaussovo funkce č. 6},
\ref{fig:Ukázka aproximace povrchu s užitím Gaussovo funkce č. 7},
\ref{fig:Ukázka aproximace povrchu s užitím Gaussovo funkce č. 8},
kde je obsažena i samotná ukázka takto navržené aproximace\footnote{Přestože
    ukázky obsahují povrchy tvořeny body v mřížce, není to vyžadováno výpočetním algoritmem.
}.
\begin{figure}[H]
    \centering
    \includegraphics[width=\textwidth]{Generated/Ukázka Gaussovo funkce 3D.pdf}
    \caption{Ukázka Gaussovo funkce $f(x,y)$}
    \label{fig:Ukázka Gaussovo funkce 3D}
\end{figure}
\begin{landscapeimagepage}
    \vspace*{\fill}
    \begin{figure}[H]
        \centering
        \begin{subfigure}{.5\textheight}
            \centering
            \includegraphics[width=\textwidth]{Generated/Ukázka aproximace s užitím Gaussovo funkce č. 1.pdf}
            \caption{Aproximace pro různé parametry Gaussovo funkce č.~1}
            \label{fig:Ukázka aproximace povrchu s užitím Gaussovo funkce č. 1}
        \end{subfigure}
        \vspace{0.5cm}
        \hspace{2.5cm}
        \begin{subfigure}{.5\textheight}
            \centering
            \includegraphics[width=\textwidth]{Generated/Ukázka aproximace s užitím Gaussovo funkce č. 2.pdf}
            \caption{Aproximace pro různé parametry Gaussovo funkce č.~2}
            \label{fig:Ukázka aproximace povrchu s užitím Gaussovo funkce č. 2}
        \end{subfigure}
        \vspace{0.5cm}
        \begin{subfigure}{.5\textheight}
            \centering
            \includegraphics[width=\textwidth]{Generated/Ukázka aproximace s užitím Gaussovo funkce č. 3.pdf}
            \caption{Aproximace pro různé parametry Gaussovo funkce č.~3}
            \label{fig:Ukázka aproximace povrchu s užitím Gaussovo funkce č. 3}
        \end{subfigure}
        \hspace{2.5cm}
        \begin{subfigure}{.5\textheight}
            \centering
            \includegraphics[width=\textwidth]{Generated/Ukázka aproximace s užitím Gaussovo funkce č. 4.pdf}
            \caption{Aproximace pro různé parametry Gaussovo funkce č.~4}
            \label{fig:Ukázka aproximace povrchu s užitím Gaussovo funkce č. 4}
        \end{subfigure}
        \caption{Ukázka aproximace povrchu č. 1 s užitím Gaussovo funkce s různými parametry}
        \label{fig:Ukázka aproximace povrchu č. 1 s užitím Gaussovo funkce s různými parametry}
    \end{figure}
    \vspace*{\fill}
\end{landscapeimagepage}

\begin{landscapeimagepage}
    \vspace*{\fill}
    \begin{figure}[H]
        \centering
        \begin{subfigure}{.5\textheight}
            \centering
            \includegraphics[width=\textwidth]{Generated/Ukázka aproximace s užitím Gaussovo funkce č. 5.pdf}
            \caption{Aproximace pro různé parametry Gaussovo funkce č.~5}
            \label{fig:Ukázka aproximace povrchu s užitím Gaussovo funkce č. 5}
        \end{subfigure}
        \vspace{0.5cm}
        \hspace{2.5cm}
        \begin{subfigure}{.5\textheight}
            \centering
            \includegraphics[width=\textwidth]{Generated/Ukázka aproximace s užitím Gaussovo funkce č. 6.pdf}
            \caption{Aproximace pro různé parametry Gaussovo funkce č.~6}
            \label{fig:Ukázka aproximace povrchu s užitím Gaussovo funkce č. 6}
        \end{subfigure}
        \vspace{0.5cm}
        \begin{subfigure}{.5\textheight}
            \centering
            \includegraphics[width=\textwidth]{Generated/Ukázka aproximace s užitím Gaussovo funkce č. 7.pdf}
            \caption{Aproximace pro různé parametry Gaussovo funkce č.~7}
            \label{fig:Ukázka aproximace povrchu s užitím Gaussovo funkce č. 7}
        \end{subfigure}
        \hspace{2.5cm}
        \begin{subfigure}{.5\textheight}
            \centering
            \includegraphics[width=\textwidth]{Generated/Ukázka aproximace s užitím Gaussovo funkce č. 8.pdf}
            \caption{Aproximace pro různé parametry Gaussovo funkce č.~8}
            \label{fig:Ukázka aproximace povrchu s užitím Gaussovo funkce č. 8}
        \end{subfigure}
        \caption{Ukázka aproximace povrchu č. 2 s užitím Gaussovo funkce s různými parametry}
        \label{fig:Ukázka aproximace povrchu č. 2 s užitím Gaussovo funkce s různými parametry}
    \end{figure}
    \vspace*{\fill}
\end{landscapeimagepage}