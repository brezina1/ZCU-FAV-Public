\subsection[NURBS interpolace povrchu]{NURBS interpolace povrchu\footcite[kapitola 9.2.5]{The_NURBS_Book}}\label{sec:NURBS interpolace povrchu}
Interpolace povrchu je podobná interpolaci 3D křivky --- máme množinu $(n + 1)
    \times (m + 1)$ bodů $\{\bm{Q}_{k,l}\}$, $k=0,\ldots,n$ a $l=0,\ldots,m$ ležících na mřížce, které
chceme interpolovat NURBS povrchem stupně $p$ a $q$, tzn.:
\begin{equation}
    \bm{Q}_{k,l} = \bm{S}(\bar{u}_k, \bar{v}_l) = \sum_{i=0}^{n}\sum_{j=0}^{m}
    N_{i,p}(\bar{u}_k)N_{j,q}(\bar{v}_l)\bm{P}_{i,j}
\end{equation}
Stejně jako u~interpolace křivky musíme vhodně zvolit parametry $\bar{u}_k$ a $\bar{v}_k$
a uzlové vektory $\bm{U}$ a $\bm{V}$.
Užitím běžných metod~(\ref{eq:chord length}) a ($\ref{eq:centripetal method}$) získáme
vektory $\hat{\bar{u}}_k$ a $\hat{\bar{v}}_k$, které musíme poté
zprůměrovat přes všechny hodnoty, tzn:
\begin{alignat}{3}
    \bar{u}_k & = \frac{1}{m + 1}\sum_{j=0}^{m}\hat{\bar{u}}_j \quad\quad & k =0,\ldots,n \\
    \bar{v}_l & = \frac{1}{n + 1}\sum_{j=0}^{n}\hat{\bar{v}}_j\quad\quad  & l =0,\ldots,m
\end{alignat}
Uzlové vektory $\bm{U}$ a $\bm{V}$ spočteme již stejně jako u~interpolace křivky,
viz~\ref{eq:uzlový vektor průměrováním}.\par
Oproti křivce, v~tomto případě $\bm{P}_{i,j}$ již není matice, ale tenzor. Tento problém můžeme
zjednodušit na interpolaci křivek postupně v~obou směrech
zafixováním jedné z~proměnných $k$ nebo $l$, tj.:
\begin{equation}
    \bm{Q}_{k,l} =\sum_{i=0}^{n}N_{i,p}(\bar{u}_k)
    \left(\sum_{j=0}^{m}N_{j,q}(\bar{v}_l)\bm{P}_{i,j}\right)
    =\sum_{i=0}^{n}N_{i,p}(\bar{u}_k)\bm{R}_{i,l}
\end{equation}
kde \begin{equation}
    \bm{R}_{i,l} = \sum_{j=0}^{m}N_{j,q}(\bar{v}_l)\bm{P}_{i,j}
\end{equation}
Tato metoda funguje pro libovolné pořadí směru interpolace křivek.
Ukázka interpolace je na obrázcích č. \ref{fig:Demo interpolace povrchu 1}, \ref{fig:Demo interpolace povrchu 2},
\ref{fig:NURBS interpolace kompenzační tabulky proudu č. 1},
\ref{fig:NURBS interpolace kompenzační tabulky proudu č. 2},
\ref{fig:NURBS interpolace kompenzační tabulky proudu č. 3},
\ref{fig:NURBS interpolace kompenzační tabulky proudu č. 4}.

\begin{imagepage}
    \begin{figure}[H]
        \centering
        \includegraphics[width=0.95\textwidth]{Generated/Demo interpolace povrchu 1.pdf}
        \caption{Porovnání algoritmů interpolace ve 2D}
        \label{fig:Demo interpolace povrchu 1}
    \end{figure}
    \begin{figure}[H]
        \centering
        \includegraphics[width=0.95\textwidth]{Generated/Demo interpolace povrchu 2.pdf}
        \caption{Porovnání algoritmů interpolace ve 3D}
        \label{fig:Demo interpolace povrchu 2}
    \end{figure}
\end{imagepage}
