\subsubsection{Cílená NURBS aproximace křivky}
Pro naše využití bychom v mnoha případech chtěli aproximovat pouze část křivky
a zbytek interpolovat, proto zavedeme tzv. cílenou aproximaci. Mějme množinu $m
    + 1$ bodů $\bm{Q}_k$ $k = 0, \ldots, m$ a aproximační interval $\bm{I}_{ap}
    \subset \bm{Q}_k$ o délce $l$.\par Na tomto intervalu $\bm{I}_{ap}$ provedeme
aproximaci ze sekce \ref{sec:NURBS aproximace křivky} s $n$ řídícími body, $n
    \le l - 2$. Tímto získáme křivku, ze které vybereme několik vhodných (např.
$2n$ ekvidistantních) bodů~$\bm{Q}_{ap}$. Interval $\bm{I}_{ap}$ v $\bm{Q}_k$
nahradíme body $\bm{Q}_{ap}$ a provedeme interpolaci. \par Tímto se i vhodně
vyhneme problému, který u aproximace nastává v případě $n \sim l$, protože si
lze snadno ohlídat počet řídících bodů $n$ k počtu aproximačních bodů $l$.
