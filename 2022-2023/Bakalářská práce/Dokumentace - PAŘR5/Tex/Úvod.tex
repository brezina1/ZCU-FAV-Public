\section{Úvod}
NURBS (Non-Uniform Rational B-Splines) spliny jsou matematickou reprezentací křivek a povrchů, nejčastěji používané v počítačové grafice a CAD softwaru. Díky svému flexibilnímu a efektivnímu způsobu definování hladkých křivek a povrchů jsou také rozšířeny v mnoha dalších odvětvích. Jedním z těchto odvětví je například robotika.
\par
V robotice lze NURBS křivky využít pro plánování hladkých pohybů manipulátoru s možností lehké obměny části pohybu za jiný bez nutnosti přepočtení celé trajektorie. V opačném případě lze NURBS křivky využít k rekonstrukci hladké trajektorie ze zašuměného záznamu pohybu pomocí aproximace.
\par
Dále je v robotice možné NURBS křivky/povrchy využít k návrhu dopředné vazby založené na nasbíraných datech. Toto se používá v případě, že zvolený regulátor návrhem nelze navrhnout tak, aby odpovídal našim představám ve všech případech --- toto může být způsobeno velkou nelinearitou řízeného systému nebo jevem, který nelze snadno modelovat/předvídat, a tím pádem by nebylo možné aplikovat dopřednou vazbu založenou na modelu systému. Příkladem takových to jevů je například časem se měnící intenzita tření (způsobena opotřebováním klíčových dílů), měnící se účinnost akčních členů (způsobena opotřebením, či kolísajícím zdrojem) a tak podobně.