\subsection[NURBS interpolace křivky]{NURBS interpolace křivky\footcite[kapitola 9.2.1]{The_NURBS_Book}}\label{section: interpolace křivky}
Mějme množinu $n + 1$ bodů $\{\bm{Q}_k\}$ $k = 0, \ldots, n$, které chceme
interpolovat NURBS křivkou stupně $p$. Pokud každému bodu $\bm{Q}_k$ přiřadíme
parametr $\bar{u}_k$, a vhodně sestrojíme uzlový vektor $\bm{U} = \{u_0, \ldots,
    u_m\}$, můžeme sestavit soustavu lineárních rovnic o rozměru $(n + 1)\times(n +
    1)$:
\begin{equation}
    \bm{Q}_k = \bm{C}(\bar{u}_k) = \sum_{i = 0}^{n}N_{i,p}(\bar{u}_k)\bm{P}_i
\end{equation}
kde řídící body $\bm{P}_i$ tvoří našich $n + 1$ neznámých.\par
Parametr $\bar{u}_k$ lze zvolit více způsoby, z nichž jsou běžné například:
\begin{itemize}
    \item ekvidistantní (equally spaced):
          \begin{alignat}{3}
              \bar{u}_0 & = 0           \quad\quad & \bar{u}_n & =1                 \\
              \bar{u}_k & = \frac{k}{n} \quad\quad & k         & = 1, \ldots, n - 1
          \end{alignat}
          Tato metoda je vhodná pro rovnoměrně rozprostřená data.
    \item délka tětivy (chord length) --- nechť $d$ je celková délka tětivy:
          \begin{equation}
              d = \sum_{k = 1}^{n}\norm{\bm{Q}_k - \bm{Q}_{k - 1}}
          \end{equation}
          potom
          \begin{equation}
              \bar{u}_0 = 0 \quad\quad \bar{u}_n = 1\\
          \end{equation}
          \begin{equation}
              \bar{u}_k = \bar{u}_{k - 1} + \frac{\norm{\bm{Q}_k - \bm{Q}_{k - 1}}}{d}
              \quad\quad k=1,\ldots,n-1
              \label{eq:chord length}
          \end{equation}
          Tato metoda je vhodná pro obecná data.
    \item dostředivá metoda (centripetal method) --- nechť $d$:
          \begin{equation}
              d = \sum_{k = 1}^{n}\sqrt{\norm{\bm{Q}_k - \bm{Q}_{k - 1}}}
          \end{equation}
          potom
          \begin{equation}
              \bar{u}_0 = 0 \quad\quad \bar{u}_n = 1\\
          \end{equation}
          \begin{equation}
              \bar{u}_k = \bar{u}_{k - 1} + \frac{\norm{\bm{Q}_k - \bm{Q}_{k - 1}}}{d}
              \quad\quad k=1,\ldots,n-1
              \label{eq:centripetal method}
          \end{equation}
          Tato metoda je vhodná pro obecná data s náhlými změnami směru.
\end{itemize}
Pro tyto metody je doporučený způsob výpočtu $\bm{U}$ metodou průměrování:
\begin{alignat}{3}
    u_0     & = \ldots = u_p = 0 \quad\quad                      & u_{m-p} & = \ldots = u_m = 1 \nonumber                             \\
    u_{j+p} & =\frac{1}{p}\sum_{i=j}^{j+p-1}\bar{u}_1 \quad\quad & j       & =1,\ldots, n -p    \label{eq:uzlový vektor průměrováním}
\end{alignat}
Nyní můžeme sestavit matici $\bm{N}$ $(n + 1) \times (n + 1)$:
\begin{equation}
    \bm{N} = \begin{bmatrix}
        N_{0,p}(\bar{u}_0)        & \cdots & N_{n+1,p}(\bar{u}_0)       \\
        \vdots                    & \ddots & \vdots                     \\
        N_{0, p}(\bar{u}_{n + 1}) & \cdots & N_{n+1,p}(\bar{u}_{n + 1})
    \end{bmatrix}
\end{equation}
Hledané řídící body $\bm{P}$ již spočteme vyřešením soustavy linárních rovnic:
\begin{align}
    \bm{P} = \bm{N}^{-1}\bm{Q}
\end{align}
Takto zavedená interpolace funguje pro libovolnou dimenzi bodů $\bm{Q}_k$.
Ukázky interpolace jsou na obrázcích č.~\ref{fig:Demo interpolace 2D} a \ref{fig:Demo interpolace 3D}.
\begin{imagepage}
    \begin{figure}[H]
        \centering
        \includegraphics[width=0.85\textwidth]{Generated/Demo interpolace křivky 2D.pdf}
        \caption{Porovnání algoritmů interpolace ve 2D}
        \label{fig:Demo interpolace 2D}
    \end{figure}
    \begin{figure}[H]
        \centering
        \includegraphics[width=0.85\textwidth]{Generated/Demo interpolace křivky 3D.pdf}
        \caption{Porovnání algoritmů interpolace ve 3D}
        \label{fig:Demo interpolace 3D}
    \end{figure}
\end{imagepage}