\subsection[NURBS křivky]{NURBS křivky \footcite[kapitola 3.2]{The_NURBS_Book}}\label{section: NURBS křivky}
Neuniformní racionální B-spline (NURBS) křivku $\bm{C}(u)$\footnote{Jedná se tedy o vektorovou funkci skalární proměnné.} definujeme následovně:
\begin{equation}
    \bm{C}(u) = \sum_{i = 0}^{n} N_{i, p}(u)\bm{P}_i\quad a\le u \le b
\end{equation}
kde
\begin{itemize}
    \item $p$ značí stupeň křivky
    \item $n$ značí počet řídících bodů
    \item $\bm{P}_i$ jsou řídící body, $\dim{\bm{P}_i} \ge 2$
    \item parametry $a$ a $b$ lze znormovat bez ztráty obecnosti --- nejčastěji se udávají hodnoty
          $a = 0$, $b = 1$, které budeme též používat
    \item $N_{i,p}$ jsou B-spline bázové funkce stupně $p$ definované
          na neperiodickém neekvidistantním uzlovém vektoru $\bm{U}$
    \item $u \in \bm{U}$
\end{itemize}
Pro uzlový vektor $\bm{U}$ platí:
\begin{equation}
    \bm{U} = \{\overbrace{\underbrace{a, \ldots, a}_{p + 1}, u_{p + 1}, \ldots, u_{m -p - 1}, \underbrace{b, \ldots, b}_{p + 1}}^{m + 1}\}
\end{equation}
kde $m = n + p + 1$.\par
Bázové funkce lze definovat rekurzivně (výhoda jednoduché implementace):
\begin{align}
    N_{i,0}(u)  & = \begin{cases}
                        1 \quad u_i \le u < u_{i + 1} \\
                        0 \quad jinak
                    \end{cases}                \nonumber         \\
    N_{i, p}(u) & = \frac{u - u_i}{u_{i + p} - u_i}N_{i,p- 1}(u)
    + \frac{u_{i + p + 1} - u}{u_{i + p + 1} - u_{i + 1}}N_{i + 1, p - 1}(u)\label{eq:bázová funkce}
\end{align}