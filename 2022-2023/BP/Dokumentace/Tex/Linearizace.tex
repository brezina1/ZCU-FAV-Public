\section{Linearizace modelu}
\begin{align}
    \bm{x} =
    \begin{bmatrix}
        x_1 \\
        x_2
    \end{bmatrix}
    = \begin{bmatrix}
          \varphi \\
          \dot{\varphi}
      \end{bmatrix}
    \\
    \bm{u} =
    \begin{bmatrix}
        u_1
    \end{bmatrix}
    = \begin{bmatrix}
          T
      \end{bmatrix}
\end{align}
Po zderivování získáme:
\begin{align}
    \bm{\dot{x}} =
    \begin{bmatrix}
        \dot{x}_1 \\
        \dot{x}_2
    \end{bmatrix}
    = \begin{bmatrix}
          \dot{\varphi} \\
          \ddot{\varphi}
      \end{bmatrix}
    =\begin{bmatrix}
         x_2 \\
         \frac{-b\cdot x_2
             - k\cdot x_1
             - g\cdot l \cdot m \cdot\sin(x_1) + u_1
         }{l^2\cdot m}
     \end{bmatrix}
\end{align}
Pokud budeme uvažovat malé úhly $x_1$, můžeme použít aproximaci $\sin(x_1) \approx x_1$:
\begin{align}
    \bm{\dot{x}} =\begin{bmatrix}
                      x_2 \\
                      \frac{-b\cdot x_2
                          - k\cdot x_1
                          - g\cdot l \cdot m \cdot x_1 + u_1
                      }{l^2\cdot m}
                  \end{bmatrix}
\end{align}
Tuto soustavu rovnic již lze snadno zapíšeme pomocí stavového modelu:
\begin{alignat}{2}
          & \bm{A} =
    \begin{bmatrix}
        0                                        & 1                     \\
        \frac{-k -g \cdot l \cdot m}{l^2\cdot m} & \frac{-b}{l^2\cdot m}
    \end{bmatrix}
    \quad &                          &
    \bm{B} =    \begin{bmatrix}
                    0 \\
                    \frac{1}{l^2 \cdot m}
                \end{bmatrix}   \\
          & \bm{C} = \begin{bmatrix}
                         1 & 0 \\
                         0 & 1
                     \end{bmatrix}
    \quad &                          &
    \bm{D} =    \begin{bmatrix}
                    0 \\
                    0
                \end{bmatrix}
\end{alignat}