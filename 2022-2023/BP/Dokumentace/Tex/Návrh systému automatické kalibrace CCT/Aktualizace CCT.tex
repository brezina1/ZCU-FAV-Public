\subsection{Aktualizace kalibrační tabulky}\label{section:aktualizace CCT}
Na obrázcích č.~\ref{fig:Původní kalibrační tabulka s aktualizačními hodnotami}, \ref{fig:Zdrojová tabulka aktualizačních hodnot}, \ref{fig:Interpolace původní kalibrační tabulky} a \ref{fig:Výsledek aktualizace CCT s užitím Gaussovo aproximace} je ukázka výsledné aktualizace kalibrační tabulky užitím Gaussovo funkce z kapitoly \nameref{section: gauss surface approximation 3D}. Jako zdrojová data pro aktualizační body, tj. nejnovější hodnoty potřebného kompenzačního proudu, jsou použity hodnoty z tabulky \nameref{fig: NURBS interpolace zpracovaných dat pomocí NURBS aproximace pro zpětný pohyb, Ids = 1 m}, kterou jsme získali analýzou záznamů pohybu manipulátoru. Byly vybrány takové body, na kterých je výsledný efekt vidět nejlépe z důvodu prezentace výsledků.
\par
Jak lze z obrázku vidět, aktualizace tabulky poskytla uspokojivé výsledky. Nicméně Gaussova aproximace v kombinaci s naší implementací sběru dat není bezvadná. Nasbírané body neobsahují hodnoty pro extrémy poloh kloubu $C_{Arc}$\footnote{Toto je způsobeno naší filtrací konstantní rychlosti, jedná se o krajní případ, který je potřeba ještě dořešit.} (0 a 1) a to znamená, že zatím není možnost aktualizace těchto krajních hodnot. 
\par
Otestování aktualizace 4D proudové kalibrační tabulky užitím algoritmu z kapitoly \nameref{section: gauss surface approximation 4D} bohužel není možné z důvodu chybějícího záznamu manuálního ovládání stroje uživatelem pro více os.
\begin{landscapeimagepage}
    \begin{figure}[H]
        \centering
        \begin{subfigure}{.5\textheight}
            \centering
            \includegraphics[width=\textwidth]{Generated/CCT před aktualizací s aktualizačními hodnotami.pdf}
            \caption{Interpolace původní kalibrační tabulky s aktualizačními hodnotami}
            \label{fig:Původní kalibrační tabulka s aktualizačními hodnotami}
        \end{subfigure}
        \vspace{0.5cm}
        \hspace{2.5cm}
        \begin{subfigure}{.5\textheight}
            \centering
            \includegraphics[width=\textwidth]{Generated/Zdrojová tabulka aktualizačních hodnot.pdf}
            \caption{Zdrojová tabulka aktualizačních hodnot --- \nameref{fig: NURBS interpolace zpracovaných dat pomocí NURBS aproximace pro zpětný pohyb, Ids = 1 m}}
            \label{fig:Zdrojová tabulka aktualizačních hodnot}
        \end{subfigure}
        \vspace{0.5cm}
        \begin{subfigure}{.5\textheight}
            \centering
            \includegraphics[width=\textwidth]{Generated/CCT před aktualizací.pdf}
            \caption{Interpolace původní kalibrační tabulky}
            \label{fig:Interpolace původní kalibrační tabulky}
        \end{subfigure}
        \hspace{2.5cm}
        \begin{subfigure}{.5\textheight}
            \centering
            \includegraphics[width=\textwidth]{Generated/Výsledek aktualizace CCT.pdf}
            \caption{Výsledek aktualizace CCT s užitím Gaussovo aproximace}
            \label{fig:Výsledek aktualizace CCT s užitím Gaussovo aproximace}
        \end{subfigure}
        \caption{Ukázka aktualizace CCT pro zpětný pohyb, $Ids = 1$ na základě měření získaného z analýzy testu odchylky proudu}
        \label{fig:Ukázka aktualizace CCT na základě měření získaného z analýzy testu odchylky proudu}
    \end{figure}
\end{landscapeimagepage}