\subsection{Analýza měřených dat}
Tato kapitola je zaměřena na sběr vhodných dat pro aktualizace kalibračních tabulek. Společnost Phillips poskytla za účelem výzkumu a vývoje autonomního kalibračního systému tyto záznamy:
\begin{enumerate}
    \item\label{item:analýza soubor jedna osa} Záznam jedné osy manuálního ovládání stroje uživatelem
    \item\label{item:analýza soubor test proudu} Záznam testů proudové odchylky na předdefinovaných trajektoriích
    % \item \todo{Záznam více os manuálního ovládání stroje uživatelem}
\end{enumerate}
\subsubsection{Analýza zaznamenaného pohybu jednoho kloubu}\label{section:Analýza záznamu jednoho kloubu}
Cílem této procedury je hledat vhodné body pro aktualizaci kalibračních tabulek. Především hledáme trajektorie s pomalým pohybem a konstantní rychlostí, protože v této oblasti se nejvíce projevují statické a kvazistatické síly, které by kalibrační tabulka měla kompenzovat, a proto jsou pro nás relevantní.\par
V této části budeme uvažovat záznam číslo \ref{item:analýza soubor jedna osa}.
Prozkoumáním záznamu lze zjistit, zda-li se daný kloub někdy pohybuje vhodnou rychlostí pro naše účely, tj.: neminimální\footnote{Při hodně malých rychlostech nejvíce působí tření, které by se značně podepsalo na měřeném proudu.}, shora omezenou\footnote{Kalibrační tabulka má největší vliv při pomalých pohybech, proto by i bylo vhodné tabulku upravovat na základě měření při takovýchto podobných rychlostech.} a konstantní.\par
Body, které splňují tyto požadavky můžeme algoritmicky odhalit následovně:
\begin{enumerate}
    \item Konstantní rychlost --- výpočtem diferencí naměřených hodnot jsme schopni sledovat absolutní změny v rychlosti v každém časovém okamžiku. Zvolením vhodného prahu\footnote{\label{footnote:volba prahu}Pro účely této práce byly zvoleny hodnoty ručně na základě naměřených hodnot}, lze takto určit kdy je rychlost téměř konstantní.
    \item Neminimální, shora omezená rychlost --- nalezneme hodnoty rychlosti, které jsou shora i zespoda omezeny vhodně zvolenými prahy\footnoteref{footnote:volba prahu}.
\end{enumerate}
Užitím pouze filtru konstantní rychlosti dostáváme body, kdy rychlost daného kloubu je skutečně téměř konstantní --- viz obrázek č.~\ref{fig:Ukázka aplikace filtru konstantní rychlosti}. Nicméně zdaleka ne všechny tyto hodnoty jsou vhodné pro aktualizaci CCT, protože vybrané hodnoty obsahují značné výkyvy v proudovém měření.
\par
Aplikací všech filtrů získáme body, které by se již daly využít pro úpravu CCT --- viz obrázek č.~\ref{fig:Ukázka aplikace všech filrů}. Je nutné dodat, že v tomto měření máme dostupné pouze 2 (poloha $C_{Arc}$, kompenzační proud) ze 4 (poloha $C_{Arc}$, $Prop$, $Ids$ a kompenzační proud) hodnot, které jsou potřeba pro získání přesné polohy v kompenzační tabulce, a tudíž na základě tohoto měření nejsme schopni přesně vyzkoušet aktualizaci CCT --- potřebná data, na základě kterých by se aktualizace dala otestovat přímo, Phillips neposkytl.
\par
Přestože nevíme 2 souřadnice, můžeme zkusit získané hodnoty dosadit do CCT. Konkrétně zkusíme dosadit nejdelší úsek vhodného pohybu nacházející se přibližně v polovině času záznamu. Z polohy lze vidět, že se jedná o zpětný pohyb, tím pádem vložíme hodnoty do našich dvou kalibračních tabulek pro zpětný pohyb. Z obrázků č.~\ref{fig:Vložení hodnot do tabulky pro zpětný pohyb pro polohu kloubu Ids = 0 m} a \ref{fig:Vložení hodnot do tabulky pro zpětný pohyb pro polohu kloubu Ids = 1 m} lze odhadnout, že poloha kloubu $Prop$ byla okolo hodnoty $0.85$ a poloha kloubu $Ids$ byla přibližně $0$. Pro tyto polohy je na obrázku č.~\ref{fig: Ukázka aktualizace CCT pomocí hodnot z měření jednoho kloubu pomocí Gaussovo aproximace} je již ukázka aktualizace CCT pomocí aproximace povrchu užitím Gaussovy funkce.

\begin{landscapeimagepage}
    \vspace*{\fill}
    \begin{figure}[H]
        \centering
        \includegraphics[width=0.85\pdfpageheight]{Generated/Aplikace filtru konstatní rychlosti.pdf}
        \caption{Ukázka aplikace filtru konstantní rychlosti}
        \label{fig:Ukázka aplikace filtru konstantní rychlosti}
    \end{figure}
    \vspace*{\fill}
\end{landscapeimagepage}

\begin{landscapeimagepage}
    \vspace*{\fill}
    \begin{figure}[H]
        \centering
        \includegraphics[width=0.85\pdfpageheight]{Generated/Aplikace všech filtrů.pdf}
        \caption{Ukázka aplikace všech filtrů}
        \label{fig:Ukázka aplikace všech filrů}
    \end{figure}
    \vspace*{\fill}
\end{landscapeimagepage}

\begin{landscapeimagepage}
    \begin{figure}[H]
        \centering
        \begin{subfigure}{.5\textheight}
            \centering
            \includegraphics[width=\textwidth]{Generated/Dosazené hodnoty kompenzačního proudu do CCT - 1.pdf}
            \caption{Vložení hodnot do tabulky pro zpětný pohyb pro polohu kloubu $Ids = 0 $m}
            \label{fig:Vložení hodnot do tabulky pro zpětný pohyb pro polohu kloubu Ids = 0 m}
        \end{subfigure}
        \hspace{2.5cm}
        \begin{subfigure}{.5\textheight}
            \centering
            \includegraphics[width=\textwidth]{Generated/Dosazené hodnoty kompenzačního proudu do CCT - 2.pdf}
            \caption{Vložení hodnot do tabulky pro zpětný pohyb pro polohu kloubu $Ids = 1 $m}
            \label{fig:Vložení hodnot do tabulky pro zpětný pohyb pro polohu kloubu Ids = 1 m}
        \end{subfigure}
        \caption{Vložení vhodných bodů ze záznamu měření jednoho kloubu}
        \label{}
    \end{figure}
    \begin{figure}[H]
        \centering
        \begin{subfigure}{.5\textheight}
            \centering
            \includegraphics[width=\textwidth]{Generated/Dosazené hodnoty kompenzačního proudu do CCT - 1 přiblíženo.pdf}
            \caption{Přiblížení dosazených hodnot}
            \label{}
        \end{subfigure}
        \hspace{2.5cm}
        \begin{subfigure}{.5\textheight}
            \centering
            \includegraphics[width=\textwidth]{Generated/Dosazené hodnoty kompenzačního proudu do CCT - 1 Gauss aprox.pdf}
            \caption{Gaussova aproximace povrchu}
            \label{}
        \end{subfigure}
        \caption{Ukázka Gaussovo aproximace povrchu pro hodnoty z měření jednoho kloubu}
        \label{fig: Ukázka aktualizace CCT pomocí hodnot z měření jednoho kloubu pomocí Gaussovo aproximace}
    \end{figure}
\end{landscapeimagepage}

