\subsection{NURBS aproximace povrchu\label{section:aproximace povrchu}}
Stejně jako u interpolace počítáme s množinou $(r + 1)\times(s + 1)$ bodů
$\bm{Q}_{k,l}$, $k=0,\ldots, r$ a $l = 0, \ldots, s$ ležících v mřížce, které chceme aproximovat
pomocí vážené metody nejmenších čtverců NURBS povrchem stupně $p$ a $q$ ve
tvaru:
\begin{align}
    S(\bar{u}_k, \bar{v}_l) = \sum_{i=0}^{r}\sum_{j=0}^{s}N_{i,p}(\bar{u}_k)N_{j,q}(\bar{v}_l)\bm{P}_{i,j}
\end{align}
Tento algoritmus interpoluje všechny rohové body a zbytek aproximuje, tj. aproximovaná plocha je tvořena body:
\begin{align}
    \begin{matrix}
        \bm{Q}_{0, 0}         & \hat{\bm{Q}}_{0, 1} & \cdots & \hat{\bm{Q}}_{0, s-1}   & \bm{Q}_{0, s}         \\
        \hat{\bm{Q}}_{1, 0}   & \ddots              &        &                         & \hat{\bm{Q}}_{1, s}   \\
        \vdots                &                     & \ddots &                         & \vdots                \\
        \hat{\bm{Q}}_{r-1, 0} &                     &        & \ddots                  & \hat{\bm{Q}}_{r-1, s} \\
        \bm{Q}_{r, 0}         & \hat{\bm{Q}}_{r, 1} & \cdots & \hat{\bm{Q}}_{r-1, s-1} & \bm{Q}_{r,s}          \\
    \end{matrix}
\end{align}
kde $\hat{\bm{Q}}_{i,j}$ je aproximovaný body $\bm{Q}_{i,j}$.\par
Stejně jako u interpolace povrchu budeme nejprve aproximovat křivky
v jednom směru a poté ve druhém. Parametry $\bar{u}_k$ a $\bar{v}_l$ a uzlové
vektory $\bm{U}$ a $\bm{V}$ získáme stejně jako
u interpolace povrchu~-~viz~\ref{sec:NURBS interpolace povrchu}.
Nejprve aproximujeme body ve směru $u$, tím získáme tenzor řídících
bodů aproximačních křivek $\bm{T}$ o rozměrech $(n + 1) \times (s + 1) \times (\dim{\bm{Q}})$:
\begin{alignat}{3}
    \bm{T}_{0, i}              & = \bm{Q}_{0, i}                                   & \quad i = 0, \ldots, s \\
    \bm{T}_{n, i}              & = \bm{Q}_{r, i}                                   & \quad i = 0, \ldots, s \\
    \bm{T}_{1 \ldots n - 1, i} & = (\bm{N}_u^T\bm{W}_u(i)\bm{N}_u)^{-1}\bm{R}_u(i) & \quad i = 0, \ldots, s
\end{alignat}
kde:
\begin{itemize}
    \item $\bm{N}_u$ je matice $(r - 1) \times (n - 1)$:
          \begin{align}
              \bm{N}_u = \begin{bmatrix}
                             N_{1,p}(\bar{u}_1)        & \cdots & N_{n-1,p}(\bar{u}_1)     \\
                             \vdots                    & \ddots & \vdots                   \\
                             N_{1, p}(\bar{u}_{r - 1}) & \cdots & N_{n-1,p}(\bar{u}_{r-1})
                         \end{bmatrix}
          \end{align}
    \item $\bm{W}_u(i)$ je matice $(r + 1)\times(s + 1)$ obsahující váhy
          bodů $\bm{Q}_{0\ldots r, 0 \ldots,s}$:
          \begin{align}
              \bm{W}_u(i)
              = \begin{bmatrix}
                    \omega^u_0(i) & 0             & \cdots & 0               \\
                    0             & \omega^u_1(i) & \cdots & 0               \\
                    \vdots        & \vdots        & \ddots & \vdots          \\
                    0             & 0             & \cdots & \omega^u_{r}(i)
                \end{bmatrix}
              = \diag(\bm{W}_{0\ldots r, i})
          \end{align}
          kde matice $\bm{W}$ obsahuje váhy jednotlivých bodů, tj.
          $\bm{W}_{i,j}$ odpovídá váze bodu~$\bm{Q}_{i,j}$
          (váhy rohových bodů nehrají roli).
    \item $\bm{R}_u(i)$ je matice $(r - 1) \times(\dim\bm{Q})$:
          \begin{equation}
              \bm{R}_u(i) =
              \begin{bmatrix}
                  N_{1,p}(\bar{u}_1)\omega^u_1(i)\bm{R}^u_1(i) + \cdots + N_{1, p}(\bar{u}_{r - 1})\omega^u_{r-1}\bm{R}^u_{r - 1}(i) \\
                  \vdots                                                                                                             \\
                  N_{n-1,p}(\bar{u}_1)\omega^u_1\bm{R}^u_1(i) + \cdots + N_{n-1, p}(\bar{u}_{r - 1})\omega^u_{r-1}\bm{R}^u_{r- 1}(i) \\
              \end{bmatrix}
          \end{equation}
          kde
          \begin{equation}
              \bm{R}^u_k(i)  = \bm{Q}_{k,i} - N_{0, p}(\bar{u}_k)\bm{Q}_{0,i} - N_{n,p}(\bar{u}_k)\bm{Q}_{r,i} \quad\quad k = 1, \ldots, r -1
          \end{equation}
\end{itemize}
Nyní stačí vypočítat řídící body aproximovaného povrchu $\bm{P}$ o stejných rozměrech
jako je $\bm{T}$.
Postup je analogický ke směru $u$, kde jako nové aproximační body použijeme
$\bm{T}$, tj.:
% ======================================================================================
\begin{alignat}{3}
    \bm{P}_{i, 0}                 & = \bm{T}_{i, 0}                                   & \quad i = 0, \ldots, n \\
    \bm{P}_{i, m}                 & = \bm{T}_{i, s}                                   & \quad i = 0, \ldots, n \\
    \bm{P}_{i, 1 \ldots m - 1, i} & = (\bm{N}_v^T\bm{W}_v(i)\bm{N}_v)^{-1}\bm{R}_v(i) & \quad i = 0, \ldots, n
\end{alignat}
kde:
\begin{itemize}
    \item $\bm{N}_v$ je matice $(s - 1) \times (m - 1)$:
          \begin{align}
              \bm{N}_u =
              \begin{bmatrix}
                  N_{1,q}(\bar{v}_1)        & \cdots & N_{m-1,q}(\bar{v}_1)     \\
                  \vdots                    & \ddots & \vdots                   \\
                  N_{1, q}(\bar{v}_{s - 1}) & \cdots & N_{m-1,q}(\bar{v}_{s-1})
              \end{bmatrix}
          \end{align}
    \item $\bm{W}_v(i)$ je diagonální matice $(n + 1)\times(s + 1)$ obsahující váhy
          bodů $\bm{T}_{0\ldots n, 0\ldots s}$:
          \begin{align}
              \bm{W}_v(i)
              = \begin{bmatrix}
                    \omega^v_0(i) & 0             & \cdots & 0               \\
                    0             & \omega^v_1(i) & \cdots & 0               \\
                    \vdots        & \vdots        & \ddots & \vdots          \\
                    0             & 0             & \cdots & \omega^v_{s}(i)
                \end{bmatrix}
              % = \imresize(W, [n + 1,])
              = \diag(\bm{W}^V_{i, 0\ldots s})
          \end{align}
          kde matice $\bm{W}^V$ je přeškálovaná matice $\bm{W}$ na
          velikost $(n+1) \times (s-1)$, stejným způsobem jako se škáluje obrázek, tj.:
          \begin{align}
              \bm{W}^V = \imresize(\bm{W}, [n + 1, s + 1])
          \end{align}
          Po provedení aproximace ve směru $u$ máme pro aproximaci ve směru $v$,
          pouze $(n+1)\times(s+1)$ bodů (namísto původních $(r+1)\times(s+1)$),
          proto je tato redukce nutná.
          Důležité je, že výsledné prvky matice v sobě nějakým způsobem nesou
          váhy pro matici původní velikosti.
          Funkce \texttt{imresize} může produkovat záporné hodnoty,
          ty ale stačí nahradit například výchozí hodnotou $1$.
    \item $\bm{R}_v(i)$ je matice $(s - 1) \times(\dim\bm{Q})$:
          \begin{equation}
              \bm{R}_v(i) =
              \begin{bmatrix}
                  N_{1,q}(\bar{v}_1)\omega^v_1(i)\bm{R}^v_1(i) + \cdots + N_{1, q}(\bar{v}_{s - 1})\omega^v_{s-1}\bm{R}^v_{s - 1}(i) \\
                  \vdots                                                                                                             \\
                  N_{m-1,q}(\bar{v}_1)\omega^v_1\bm{R}^v_1(i) + \cdots + N_{m-1, q}(\bar{v}_{s - 1})\omega^v_{s-1}\bm{R}^v_{s- 1}(i) \\
              \end{bmatrix}
          \end{equation}
          kde
          \begin{equation}
              \bm{R}^v_l(i)  = \bm{T}_{i,l} - N_{0, q}(\bar{v}_l)\bm{T}_{i,0} - N_{m,q}(\bar{v}_l)\bm{T}_{i, s} \quad\quad l = 1, \ldots, s -1
          \end{equation}
\end{itemize}
Ukázka různých aproximací jsou na obrázcích č. \ref{fig:Demo aproximace povrchu č. 1},
\ref{fig:Demo aproximace povrchu č. 2},
\ref{fig:Demo aproximace povrchu pro různé váhy č. 1},
\ref{fig:Demo aproximace povrchu pro různé váhy č. 2},
\ref{fig:Demo aproximace povrchu pro různé váhy č. 3} a
\ref{fig:Demo aproximace povrchu pro různé váhy č. 4}.

\begin{imagepage}
    \begin{figure}[H]
        \centering
        \includegraphics[width=0.95\textwidth]{Generated/Demo aproximace povrchu 1.pdf}
        \caption{Aproximace povrchu metodou nejmenších čtverců č. 1}
        \label{fig:Demo aproximace povrchu č. 1}
    \end{figure}
    \begin{figure}[H]
        \centering
        \includegraphics[width=0.95\textwidth]{Generated/Demo aproximace povrchu 2.pdf}
        \caption{Aproximace povrchu metodou nejmenších čtverců č. 2}
        \label{fig:Demo aproximace povrchu č. 2}
    \end{figure}
\end{imagepage}

\begin{imagepage}
    \begin{figure}[H]
        \centering
        \includegraphics[width=0.95\textwidth]{Generated/Ukázka aproximace povrchu pro různé váhy bodů 1.pdf}
        \caption{Aproximace povrchu pro různé váhy č. 1}
        \label{fig:Demo aproximace povrchu pro různé váhy č. 1}
    \end{figure}
    \begin{figure}[H]
        \centering
        \includegraphics[width=0.95\textwidth]{Generated/Ukázka aproximace povrchu pro různé váhy bodů 2.pdf}
        \caption{Aproximace povrchu pro různé váhy č. 2}
        \label{fig:Demo aproximace povrchu pro různé váhy č. 2}
    \end{figure}
\end{imagepage}

\begin{imagepage}
    \begin{figure}[H]
        \centering
        \includegraphics[width=0.95\textwidth]{Generated/Ukázka aproximace povrchu pro různé váhy bodů 3.pdf}
        \caption{Aproximace povrchu pro různé váhy č. 3}
        \label{fig:Demo aproximace povrchu pro různé váhy č. 3}
    \end{figure}
    \begin{figure}[H]
        \centering
        \includegraphics[width=0.95\textwidth]{Generated/Ukázka aproximace povrchu pro různé váhy bodů 4.pdf}
        \caption{Aproximace povrchu pro různé váhy č. 4}
        \label{fig:Demo aproximace povrchu pro různé váhy č. 4}
    \end{figure}
\end{imagepage}