\subsection{NURBS aproximace křivky\label{sec:NURBS aproximace křivky}}
Mějme množinu $m + 1$ bodů $\bm{Q_k}$ ${k = 0, \ldots, m}$ a hledáme
křivku stupně $p$ ve tvaru:
\begin{equation}
    \bm{C}(u) = \sum_{i = 0}^{n}N_{i,p}(u)\bm{P}_i\quad u\in[0,1], \quad\quad n \ge p \ge 1
\end{equation}
pro kterou platí:
\begin{itemize}
    \item krajní body interpoluje, tj.: $\bm{Q}_0 = \bm{C}(0)$, $\bm{Q}_m = \bm{C}(1)$
    \item zbytek bodů aproximuje váženou metodou nejmenších čtverců:
          \begin{align}
              \bm{P} = (\bm{N}^T \bm{W} \bm{N})^{-1}\bm{R}
          \end{align}
          kde
          \begin{itemize}
              \item $\bm{N}$ je matice $(m - 1) \times (n - 1)$:
                    \begin{equation}
                        \bm{N} = \begin{bmatrix}
                            N_{1,p}(\bar{u}_1)        & \cdots & N_{n-1,p}(\bar{u}_1)     \\
                            \vdots                    & \ddots & \vdots                   \\
                            N_{1, p}(\bar{u}_{m - 1}) & \cdots & N_{n-1,p}(\bar{u}_{m-1})
                        \end{bmatrix}
                    \end{equation}
              \item $\bm{R}$ je matice $(n - 1)\times(\dim\bm{Q})$:
                    \begin{equation}
                        \bm{R} =
                        \begin{bmatrix}
                            N_{1,p}(\bar{u}_1)\omega_1\bm{R}_1 + \cdots + N_{1, p}(\bar{u}_{m - 1})\omega_{m-1}\bm{R}_{m - 1}     \\
                            \vdots                                                                                                \\
                            N_{n-1,p}(\bar{u}_1)\omega_1\bm{R}_1 + \cdots + N_{n-1, p}(\bar{u}_{m - 1})\omega_{m-1}\bm{R}_{m - 1} \\
                        \end{bmatrix} \\
                    \end{equation}
                    kde
                    \begin{equation}
                        \bm{R}_k  = \bm{Q}_k - N_{0, p}(\bar{u}_k)\bm{Q}_0 - N_{n,p}(\bar{u}_k)\bm{Q}_m \quad\quad k = 1, \ldots, m -1
                    \end{equation}
              \item $\bm{W}$ je diagonální matice $(m - 1) \times (m - 1)$\footnote{Váhy
                        krajních bodů nehrají roli, protože tyto body jsou interpolovány}
                    s váhami jednotlivých bodů:
                    \begin{equation}
                        \bm{W}  = \begin{bmatrix}
                            \omega_1 & 0        & \cdots & 0            \\
                            0        & \omega_2 & \cdots & 0            \\
                            \vdots   & \vdots   & \ddots & \vdots       \\
                            0        & 0        & \cdots & \omega_{m-1}
                        \end{bmatrix}
                    \end{equation}
              \item $\bm{P}$ je matice hledaných řídících bodů $(n + 1)\times(\dim\bm{Q})$

          \end{itemize}
\end{itemize}
Opět musíme vhodně zvolit parametry $\bar{u}_k$ a uzlový vektor $\bm{U}$. Parametr $\bar{u}$
můžeme vypočítat pomocí předpisu \ref{eq:chord length}.
\par Prvky uzlového vektoru poté vypočteme následovně:
\begin{equation}
    d = \frac{m + 1}{n - p + 1}
\end{equation}
\begin{alignat}{3}
    i         & = \lfloor j \cdot d \rfloor                                 & \quad\quad \alpha    & = j \cdot d - i              \\
    u_{p + j} & = (1 - \alpha)\cdot\bar{u}_{i - 1} + \alpha \cdot \bar{u}_i & \quad\quad j         & = 1, \ldots, n - p           \\
    u_0       & = \ldots = u_p = 0 \quad\quad                               & \quad\quad u_{n + 1} & = \ldots = u_{n + p + 1} = 1
\end{alignat}

Takto navržená aproximace je citlivá na poměr počtu řídících bodů a počtu
aproximačních bodů tj.: $r = \frac{n}{m + 1}$. Jak se $r$ blíží k $1$, tím je
větší šance vzniku velkých kmitů, obzvlášť v případě zašuměných dat --- viz
obrázky č. \ref{fig:Demo aproximace křivky 1}, \ref{fig:Demo aproximace křivky
    2}, \ref{fig:Demo aproximace křivky 3}, \ref{fig:Demo aproximace křivky 4}.\par
Ideálním případem pro tento algoritmus je když platí $m \gg n$ --- viz obrázek č.
\ref{fig:Demo aproximace křivky 5}. Obrázky č. \ref{fig:Demo aproximace pro
    různé váhy 1}, \ref{fig:Demo aproximace pro různé váhy 2}, \ref{fig:Demo
    aproximace pro různé váhy 3} dále ukazují chování algoritmu pro různé váhy
bodů.
\begin{imagepage}
    \begin{figure}[H]
        \centering
        \includegraphics[width=0.85\textwidth]{Generated/Demo aproximace křivky 1.pdf}
        \caption{Aproximace křivky metodou nejmenších čtverců pro různé parametry}
        \label{fig:Demo aproximace křivky 1}
    \end{figure}
    \begin{figure}[H]
        \centering
        \includegraphics[width=0.85\textwidth]{Generated/Demo aproximace křivky 2.pdf}
        \caption{Aproximace křivky metodou nejmenších čtverců pro různé parametry}
        \label{fig:Demo aproximace křivky 2}
    \end{figure}
\end{imagepage}
\begin{imagepage}
    \begin{figure}[H]
        \centering
        \includegraphics[width=0.85\textwidth]{Generated/Demo aproximace křivky 3.pdf}
        \caption{Aproximace křivky metodou nejmenších čtverců pro různé parametry}
        \label{fig:Demo aproximace křivky 3}
    \end{figure}
    \begin{figure}[H]
        \centering
        \includegraphics[width=0.85\textwidth]{Generated/Demo aproximace křivky 4.pdf}
        \caption{Aproximace křivky metodou nejmenších čtverců pro různé parametry}
        \label{fig:Demo aproximace křivky 4}
    \end{figure}
\end{imagepage}
\begin{imagepage}
    \begin{figure}[H]
        \centering
        \includegraphics[width=0.85\textwidth]{Generated/Demo aproximace křivky 5.pdf}
        \caption{Aproximace křivky metodou nejmenších čtverců pro různé parametry}
        \label{fig:Demo aproximace křivky 5}
    \end{figure}
    \begin{figure}[H]
        \centering
        \includegraphics[width=0.85\textwidth]{Generated/Ukázka aproximace pro různé váhy bodů 1.pdf}
        \caption{Aproximace křivky metodou nejmenších čtverců pro různé váhy bodů}
        \label{fig:Demo aproximace pro různé váhy 1}
    \end{figure}
\end{imagepage}

\begin{imagepage}
    \begin{figure}[H]
        \centering
        \includegraphics[width=0.85\textwidth]{Generated/Ukázka aproximace pro různé váhy bodů 2.pdf}
        \caption{Aproximace křivky metodou nejmenších čtverců pro různé váhy bodů}
        \label{fig:Demo aproximace pro různé váhy 2}
    \end{figure}
    \begin{figure}[H]
        \centering
        \includegraphics[width=0.85\textwidth]{Generated/Ukázka aproximace pro různé váhy bodů 3.pdf}
        \caption{Aproximace křivky metodou nejmenších čtverců pro různé váhy bodů}
        \label{fig:Demo aproximace pro různé váhy 3}
    \end{figure}
\end{imagepage}