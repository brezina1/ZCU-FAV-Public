\subsection{Struktura řídícího systému}
Kompenzace proudu pohonu pro osu $C_{Arc}$ je závislá na polohách všech tří kloubů
($C_{Arc}$, $Prop$, $Ids$) a také na směru pohybu\footnote{Kompenzace proudu pohonu pro zbylé osy také pravděpodobně závisí na ostatních osách, nicméně jsou dostupná data pouze pro kompenzační tabulku osy $C_{Arc}$}.

% Pro zatím uvažujeme pouze dvě možné polohy kloubu $Ids$ a dva možné směry pohybu (dopředný a zpětný), což nám umožní závislost zaznamenávat pomocí čtyř 2D tabulek. Současná implementace kompenzačních tabulek je postavena právě na tomto rozdělení na 4 tabulky pro každý kloub. 
V současném řídícím algoritmu jsou uvažovány pouze dvě možné polohy kloubu $Ids$ a dva možné směry pohybu (dopředný a zpětný), což nám umožní vizualizovat proudovou kalibrační tabulku prostřednictvím čtyř 3D grafů povrchu.
Vizualizovaná data z těchto 4 tabulek pro kloub $C_{Arc}$ lze vidět na obrázcích č. \ref{fig:kompenzace proudu reálná data 1}, \ref{fig:kompenzace proudu reálná data 2},
\ref{fig:kompenzace proudu reálná data 3}, \ref{fig:kompenzace proudu reálná data 4}.

\subsubsection{Schéma regulační smyčky}
K zajištění přesného a plynulého pohybu regulační smyčka obsahuje 3 hlavní kompenzátory:
\begin{itemize}
    \item $C_{PID}$ --- zpětnovazební PID regulátor \todo{dohodit lepší popis}
    \item $C_{FF}$ --- přímovazební modelově orientovaný regulátor
    \item $CCT$ --- přímovazební datově orientovaný regulátor v podobě kompenzační tabulky
\end{itemize}
Schéma regulační smyčky pro polohu kloubu $C_{Arc}$ vypadá přibližně takto:
\tikzset
{
    block/.style = {draw, fill=white, rectangle, minimum height=2em, minimum width=3em},
    sum/.style n args={4}{draw, fill=white,
            circle, minimum width=6mm, node distance=2cm, path picture={
                    \node at ($(path picture bounding box.south)+(0,0.13)$) {\tiny{#1}};
                    \node at ($(path picture bounding box.west)+(0.13,0)$) {\tiny{#2}}; \node at
                    ($(path picture bounding box.north)+(0,-0.13)$) {\tiny{#3}}; \node at ($(path
                    picture bounding box.east)+(-0.13,0)$) {\tiny{#4}};
                }
        },
    input/.style = {coordinate},
    output/.style= {coordinate},
}

\begin{figure}[H]
    \centering
    \begin{tikzpicture}[auto, node distance=1.5cm,>=latex']
        \shorthandoff{-}
        % \coordinate[] (reference_spacer) {};
        \node[input, label={$r$ \tiny{$[\degree]$}}] (reference) {};
        \node[sum={--}{+}{}{}, right of=reference] (zpetna_vazba) {};

        \node[block, right of=zpetna_vazba, label=below:\tiny{Zpětnovazební reg.}] (zpetnovazebni_reg) {$C_{\text{\tiny{PID}}}$};
        \node[block, above of=zpetnovazebni_reg, label=below:\tiny{Přímovazební reg.}] (primovazebni_reg) {$C_{\text{\tiny{FF}}}$};
        \node[block, above of=primovazebni_reg, label=below:\tiny{Kompenzační tabulka}] (kompenzacni_tab) {$CCT$};
        \node[input, left of=kompenzacni_tab] (other_axis) {};
        \node[input, left of=kompenzacni_tab] (other_other_axis) {};
        \node[sum={}{+}{+}{}, right of=zpetnovazebni_reg] (reg_sum) {};
        \node[block, right of=reg_sum, xshift=1cm, label=below:\tiny{Elektropohon}] (process) {P};
        \node[output, right of=process] (output) {};
        \coordinate [below of=reg_sum,yshift = 0.5cm] (below_reg_sum) {};

        \draw[->] (reference) -- (zpetna_vazba) node [midway, anchor=north] (r) {};
        \node[above of=r] (above_r) {};
        \node[sum={}{+}{+}{}, right of=primovazebni_reg] (right_of_primovazebni_reg) {};
        \node[sum={}{}{}{}, draw=none, right of=kompenzacni_tab] (right_of_kompenzacni_tab) {};
        % \draw[->] (r) |- (kompenzacni_tab);

        \draw[->] (other_axis) node[xshift=-5mm] () {\tiny{$Prop$}}  |- (kompenzacni_tab);
        \begin{scope}[transform canvas={yshift=+3mm}]
            \draw[->] (other_other_axis) node[xshift=-5mm] () {\tiny{$Ids$}} |- (kompenzacni_tab);
        \end{scope}

        \begin{scope}[transform canvas={yshift=-3mm}]
            \draw[->] (above_r) |- (kompenzacni_tab);
        \end{scope}

        \path (kompenzacni_tab) -- node [midway] () {\tiny{$[mA]$}} (right_of_kompenzacni_tab);
        \draw[->] (primovazebni_reg) -- node [midway] () {\tiny{$[A]$}} (right_of_primovazebni_reg);

        \draw[->] (r) |- (primovazebni_reg);
        \draw[->] (r) |- (zpetna_vazba);

        \draw[->] (process) -- node [midway] (y) {$y$ \tiny{$[\degree]$}} (output);
        \draw[->] (reg_sum) -- node [midway] (u) {$u$ \tiny{$[A]$}} (process);
        \draw[->] (zpetnovazebni_reg) -- node [midway] () {\tiny{$[A]$}} (reg_sum);
        \draw[->] (right_of_primovazebni_reg) --  (reg_sum);
        \draw[->] (kompenzacni_tab) -| (right_of_primovazebni_reg);
        \draw[->] (below_reg_sum) -| (zpetna_vazba);
        \draw[-] (y) |- (below_reg_sum);
        \draw[->] (zpetna_vazba) -- (zpetnovazebni_reg);
        \shorthandon{-}
    \end{tikzpicture}
    \caption{Schéma regulační smyčky CT pro jednu osu --- $C_{Arc}$}
\end{figure}
Oba regulátory běžně operují v řádu ampér, zatímco kompenzační tabulka v řádu stovek miliampér.
To naznačuje, že tabulka má největší vliv při malých rychlostech (např. při precizních pohybech), kdy výstupy regulátorů jsou v podobném rozsahu jako tabulka.
% Cílem práce je takto uspořádané kompenzační tabulky vhodně automaticky aktualizovat
% na základě naměřených hodnot při běžné činnosti robota.


\begin{landscapeimagepage}
    \begin{figure}[H]
        \centering
        \begin{subfigure}{.5\textheight}
            \centering
            \includegraphics[width=\textwidth]{Generated/Plot kompenzace proudu - reálná data 1.pdf}
            \caption{Vizualizace kompenzační tabulky pro osu $x$ --- dopředný pohyb, $Ids = 0$}
            \label{fig:kompenzace proudu reálná data 1}
        \end{subfigure}
        \vspace{0.5cm}
        \hspace{2.5cm}
        \begin{subfigure}{.5\textheight}
            \centering
            \includegraphics[width=\textwidth]{Generated/Plot kompenzace proudu - reálná data 2.pdf}
            \caption{Vizualizace kompenzační tabulky pro osu $x$ --- dopředný pohyb, $Ids = 1$}
            \label{fig:kompenzace proudu reálná data 2}
        \end{subfigure}
        \vspace{0.5cm}
        \begin{subfigure}{.5\textheight}
            \centering
                    \includegraphics[width=\textwidth]{Generated/Plot kompenzace proudu - reálná data 3.pdf}
                    \caption{Vizualizace kompenzační tabulky pro osu $x$ --- zpětný pohyb, $Ids = 0$}
                    \label{fig:kompenzace proudu reálná data 3}
        \end{subfigure}
        \hspace{2.5cm}
        \begin{subfigure}{.5\textheight}
                \centering
            \includegraphics[width=\textwidth]{Generated/Plot kompenzace proudu - reálná data 4.pdf}
            \caption{Vizualizace kompenzační tabulky pro osu $x$ --- dopředný pohyb, $Ids = 1$}
            \label{fig:kompenzace proudu reálná data 4}
        \end{subfigure}
        \caption{Vizualizace kompenzační tabulek pro osu $C_{Arc}$}
        \label{fig: Vizualizace kompenzační tabulek pro osu CArc}
    \end{figure}
\end{landscapeimagepage}

% \begin{imagepage}
%     \begin{figure}[H]
%         \centering
%         \includegraphics[width=\textwidth]{Generated/Plot kompenzace proudu - reálná data 1.pdf}
%         \caption{Vizualizace kompenzační tabulky pro osu $x$ - dopředný pohyb, $z = 895$mm}
%         \label{fig:kompenzace proudu reálná data 1}
%     \end{figure}
%     \begin{figure}[H]
%         \centering
%         \includegraphics[width=\textwidth]{Generated/Plot kompenzace proudu - reálná data 2.pdf}
%         \caption{Vizualizace kompenzační tabulky pro osu $x$ - dopředný pohyb, $z = 1170$mm}
%         \label{fig:kompenzace proudu reálná data 2}
%     \end{figure}
% \end{imagepage}

% \begin{imagepage}
%     \begin{figure}[H]
%         \centering
%         \includegraphics[width=\textwidth]{Generated/Plot kompenzace proudu - reálná data 3.pdf}
%         \caption{Vizualizace kompenzační tabulky pro osu $x$ - zpětný pohyb, $z = 895$mm}
%         \label{fig:kompenzace proudu reálná data 3}
%     \end{figure}
%     \begin{figure}[H]
%         \centering
%         \includegraphics[width=\textwidth]{Generated/Plot kompenzace proudu - reálná data 4.pdf}
%         \caption{Vizualizace kompenzační tabulky pro osu $x$ - dopředný pohyb, $z = 1170$mm}
%         \label{fig:kompenzace proudu reálná data 4}
%     \end{figure}
% \end{imagepage}